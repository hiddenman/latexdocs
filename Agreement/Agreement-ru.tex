\documentclass[pdftex,a4paper,10pt,oneside]{scrartcl}
\usepackage[left=2cm,right=2cm,top=2cm,bottom=3cm,bindingoffset=0cm]{geometry}
\usepackage{cmap} % чтобы работал поиск по PDF
\usepackage[T2A]{fontenc}
\usepackage[utf8]{inputenc}

% our vars
\def\ipxpname{IPXP Europe s.r.o.}
\def\ipxpaddress{Bubenec, Ceskoslovenske armady 371/11, PSC 160 00, Prague 6, Czech Republic}


% do not enlarge space after punctuation marks
\frenchspacing
% make bold font better
\renewcommand{\bfdefault}{b}
% we can hyphen and left two symbols in Russian
\righthyphenmin=2

% use cm-super
%\renewcommand{\rmdefault}{cmr}
%\renewcommand{\sfdefault}{cmss}
%\renewcommand{\ttdefault}{cmtt}

% use pscyr
\usepackage{pscyr}
%\renewcommand{\rmdefault}{fjn}
%\renewcommand{\sfdefault}{fma}
%\renewcommand{\ttdefault}{fcr}
\renewcommand{\rmdefault}{ftm}
\renewcommand{\sfdefault}{far}
\renewcommand{\ttdefault}{fcr}
% font for PDF forms
%\renewcommand*{\familydefault}{cmr}
%\renewcommand*{\familydefault}{fjn}
\renewcommand*{\familydefault}{ftm}


% titlesec etc.
\usepackage[bf,small,compact]{titlesec}
\usepackage{indentfirst} % first paragraph indent

% format of section: X.
\makeatletter
\renewcommand*{\@seccntformat}[1]{%
   \csname the#1\endcsname.\;
}
% dot at the TOC
\renewcommand\numberline[1]{\hb@xt@\@tempdima{#1.\hfil}}
\makeatother


\lccode`\-=`\-
\defaulthyphenchar=127


\usepackage[english,russian]{babel}
% for date formatting
\usepackage{datetime}
\renewcommand{\timeseparator}{:}
\renewcommand{\dateseparator}{.}
% end datetime
\usepackage{url} % urling
\usepackage{fancyhdr} % управление колонтитулами
\usepackage{amssymb} % математические символы. пригодятся в пакете ntheorem
\usepackage{amsmath} % математические символы. пригодятся в пакете ntheorem
\usepackage[amsthm,amsmath,thmmarks,hyperref]{ntheorem} % поддержка точек после номера теорем
\usepackage[warn]{mathtext}
\usepackage{srcltx}
\usepackage{textcomp}
\usepackage{wrapfig}
\usepackage{tabularx}
\usepackage{afterpage}
\usepackage{ccaption} % подписи к рисункам с точкой и пр.
\usepackage{color}
\usepackage{makeidx}
\usepackage{xspace}
\usepackage{wasysym}
\usepackage{enumerate}
\usepackage{enumitem}    % Enhancing the list environments
\usepackage{underscore}
\usepackage{bigstrut}

\usepackage{calc}

\usepackage[pdftex]{graphicx}
\usepackage{epstopdf}
\pdfcompresslevel=9 % сжимать PDF


\newcommand{\bs}{\symbol{'134}}
\newcommand{\grad}{\ensuremath{{}^{\circ}}\xspace}

\deffootnote[2.5em]{1.5em}{1em}{\textsuperscript{\thefootnotemark}}



\addto\captionsrussian{\def\contentsname{\textbf{Оглавление.}}}
\addto\captionsrussian{\def\refname{\textbf{Литература.}}}
                        % Наклонные русские буквы в формулах при установленном
                        % пакете pscyr
                        % подпись к рисунку в виде цифры с точкой (и пробелом!),
                        % а не сдвоеточием требует пакета ccaption

\captiondelim{. }
                        % Добавить это перед begin{thebibliography} для
                        % нумерации списка литературы в виде цифры с точкой
\makeatletter
\renewcommand{\@biblabel}[1]{#1.}
\makeatother
                        % определение теорем, необходим пакет ntheorem
\newtheorem{theo}{Теорема}
                        % Конец преамбулы

% enumerate style
\renewcommand\theenumi{\arabic{enumi}}
\renewcommand\labelenumi{\theenumi.}
\renewcommand\theenumii{\arabic{enumii}}
\renewcommand\labelenumii{\theenumi.\theenumii.}
\renewcommand\theenumiii{\arabic{enumiii}}
\renewcommand\labelenumiii{\labelenumii\theenumiii.}
\renewcommand\theenumiv{\arabic{enumiv}}
\renewcommand\labelenumiv{\labelenumiii\theenumiv.}

\makeatletter \renewcommand\p@enumii{\theenumi.} \makeatother

% some variables for better formatting
\clubpenalty=1000
\widowpenalty=1000
\tolerance=300

% bullets
\renewcommand{\labelitemi}{$\bullet$}
\renewcommand{\labelitemii}{$\cdot$}
\renewcommand{\labelitemiii}{$\diamond$}
\renewcommand{\labelitemiv}{$\ast$}


%%%%%%%%%%%%%%%%%%%%%%%%%%%%%%%%%%%%%%%%%%%%%%%%%%
% fancyhdr
\usepackage{fancyhdr}
\pagestyle{fancy}

% for page number
\usepackage{lastpage}

\newsavebox{\ipxplogo}
\savebox{\ipxplogo}{\noindent\includegraphics[keepaspectratio]{logo-ipxp}}

\renewcommand{\headrulewidth}{0.4pt}
\renewcommand{\footrulewidth}{0.4pt}

\chead{\thepage}
%\chead{\thepage\ of \pageref{LastPage}}
\rhead{\textbf{Договор}}

\cfoot{\usebox{\ipxplogo}}
\lfoot{\parbox{5.5cm}{\rule{5.5cm}{0.4pt}\\ \TextField[name=ipxpname,charsize=10pt,width=5.5cm,height=1em,align=1,borderwidth=0,bordercolor={1 1 1},value=\ipxpname,readonly=true]{}}}
\rfoot{\parbox{5.5cm}{\rule{5.5cm}{0.4pt}\\ \TextField[name=clntname,charsize=10pt,width=5.5cm,height=1em,align=1,borderwidth=0,bordercolor={1 1 1}]{}}}

% for the first page
\fancypagestyle{agreement-noheader}{%
\fancyhf{} % clear all header and footer fields
\lfoot{\parbox{5.5cm}{\rule{5.5cm}{0.4pt}\\ \TextField[name=ipxpname,charsize=10pt,width=5.5cm,height=1em,align=1,borderwidth=0,bordercolor={1 1 1},value=\ipxpname,readonly=true]{}}}
\cfoot{\usebox{\ipxplogo}}
\rfoot{\parbox{5.5cm}{\rule{5.5cm}{0.4pt}\\ \TextField[name=clntname,charsize=10pt,width=5.5cm,height=1em,align=1,borderwidth=0,bordercolor={1 1 1}]{}}}
\renewcommand{\headrulewidth}{0.0pt}
}

\fancypagestyle{agreement-nofooter}{%
\fancyhf{} % clear all header and footer fields
\cfoot{\usebox{\ipxplogo}}
}

% for the first page
\fancypagestyle{appendix}{%
\fancyhf{} % clear all header and footer fields
\chead{\thepage}
%\chead{\thepage\ of \pageref{LastPage}}
\rhead{\textbf{Приложение А}}
\lfoot{}
\lfoot{\parbox{5.5cm}{\rule{5.5cm}{0.4pt}\\ \TextField[name=ipxpname,charsize=10pt,width=5.5cm,height=1em,align=1,borderwidth=0,bordercolor={1 1 1},value=\ipxpname,readonly=true]{}}}
\cfoot{\usebox{\ipxplogo}}
\rfoot{\parbox{5.5cm}{\rule{5.5cm}{0.4pt}\\ \TextField[name=clntname,charsize=10pt,width=5.5cm,height=1em,align=1,borderwidth=0,bordercolor={1 1 1}]{}}}
}

% for the first page
\fancypagestyle{appendix-noheader}{%
\fancyhf{} % clear all header and footer fields
\renewcommand{\headrulewidth}{0.0pt}
\lfoot{\parbox{5.5cm}{\rule{5.5cm}{0.4pt}\\ \TextField[name=ipxpname,charsize=10pt,width=5.5cm,height=1em,align=1,borderwidth=0,bordercolor={1 1 1},value=\ipxpname,readonly=true]{}}}
\cfoot{\usebox{\ipxplogo}}
\rfoot{\parbox{5.5cm}{\rule{5.5cm}{0.4pt}\\ \TextField[name=clntname,charsize=10pt,width=5.5cm,height=1em,align=1,borderwidth=0,bordercolor={1 1 1}]{}}}
}

\fancypagestyle{appendix-nofooter}{%
\fancyhf{} % clear all header and footer fields
\cfoot{\usebox{\ipxplogo}}
}

\thispagestyle{agreement-noheader}
% end fancyhdr
%%%%%%%%%%%%%%%%%%%%%%%%%%%%%%%%%%%%%%%%%%%%%%%%%%



% special macro for underscores
%\def\rp{\raisebox{-1pt}{\rule{1cm}{.4pt}}\hspace{0pt}}
\def\rp{\raisebox{0pt}{\rule{0.5cm}{.4pt}}\penalty100}

\usepackage[pdftex,a4paper,final,bookmarks=true,bookmarksopen,bookmarksnumbered,plainpages=false,colorlinks=false, unicode,pdfpagelabels, plainpages=false, pdfstartview=FitH]{hyperref}
% JavaScript
\usepackage[pdftex]{insdljs}
\hypersetup{
 pdftitle={Договор о телекоммуникационных услугах,Версия 1.0, \pdfdate},
 pdfauthor=IPXP Europe s.r.o.,
 pdfsubject={Договор о телекоммуникационных услугах},
 pdfkeywords={соглашение, договор, приложение}
}


\begin{insDLJS}[docOpen]{agreementru}{JavaScript}
function docOpen()
{
   this.getField("contractdate").textFont = font.Times;
   this.getField("contractdate").userName = "Enter contract date";

   this.getField("contractnum").textFont = font.Times;
   this.getField("contractnum").userName = "Enter contract number";

   this.getField("ipxpname").textFont = font.TimesB;
   this.getField("ipxpname").userName = "Enter company name";

   this.getField("ipxpmanager").textFont = font.TimesI;
   this.getField("ipxpmanager").userName = "Enter manager name and e-mail";
   
   this.getField("clntname").textFont = font.TimesB;
   this.getField("clntname").userName = "Enter company name";
   
   this.getField("clntdirname").textFont = font.TimesI;
   this.getField("clntdirname").userName = "Enter director name";

   this.getField("clntcountry").textFont = font.TimesI;
   this.getField("clntcountry").userName = "Enter country name";

   this.getField("clntregnum").textFont = font.TimesI;
   this.getField("clntregnum").userName = "Enter registration number";

   this.getField("clntlegaladdr").textFont = font.TimesI;
   this.getField("clntlegaladdr").userName = "Enter legal address";

   this.getField("clntlegaladdrnext").textFont = font.TimesI;
   this.getField("clntlegaladdrnext").userName = "Enter legal address";

   this.getField("clntcorraddr").textFont = font.TimesI;
   this.getField("clntcorraddr").userName = "Enter address for correspondence";

   this.getField("clntcorraddrnext").textFont = font.TimesI;
   this.getField("clntcorraddrnext").userName = "Enter address for correspondence";

   this.getField("clntmanager").textFont = font.TimesI;
   this.getField("clntmanager").userName = "Enter manager name and e-mail";

   this.getField("clntmanagernext").textFont = font.TimesI;
   this.getField("clntmanagernext").userName = "Enter manager name and e-mail";
   
   this.getField("clntphone").textFont = font.TimesI;
   this.getField("clntphone").userName = "Enter phone number in E.164 format";

   this.getField("clntphonenext").textFont = font.TimesI;
   this.getField("clntphonenext").userName = "Enter phone number in E.164 format";

   this.getField("clntfax").textFont = font.TimesI;
   this.getField("clntfax").userName = "Enter fax number in E.164 format";

   this.getField("clntfaxnext").textFont = font.TimesI;
   this.getField("clntfaxnext").userName = "Enter fax number in E.164 format";
   
   this.getField("clntattorneyprintname").textFont = font.TimesI;
   this.getField("clntattorneyprintname").userName = "Enter attorney print name";

   this.getField("clntattorneyposition").textFont = font.TimesI;
   this.getField("clntattorneyposition").userName = "Enter attorney position";

   this.getField("clntgeneralemail").textFont = font.TimesB;
   this.getField("clntgeneralemail").userName = "Enter email for general support";

   this.getField("clntgeneralemailnext").textFont = font.TimesB;
   this.getField("clntgeneralemailnext").userName = "Enter email for general support";

   this.getField("clntnocemail").textFont = font.TimesB;
   this.getField("clntnocemail").userName = "Enter email for technical support";

   this.getField("clntnocemailnext").textFont = font.TimesB;
   this.getField("clntnocemailnext").userName = "Enter email for technical support";

   this.getField("clntqaemail").textFont = font.TimesB;
   this.getField("clntqaemail").userName = "Enter email for quality assurance";

   this.getField("clntqaemailnext").textFont = font.TimesB;
   this.getField("clntqaemailnext").userName = "Enter email for quality assurance";
   
   this.getField("clntratesemail").textFont = font.TimesB;
   this.getField("clntratesemail").userName = "Enter email for rates notifications";

   this.getField("clntratesemailnext").textFont = font.TimesB;
   this.getField("clntratesemailnext").userName = "Enter email for rates notifications";

   this.getField("clntagreementsemail").textFont = font.TimesB;
   this.getField("clntagreementsemail").userName = "Enter email for agreements and appendices";

   this.getField("clntagreementsemailnext").textFont = font.TimesB;
   this.getField("clntagreementsemailnext").userName = "Enter email for agreements and appendices";

   this.getField("clntinvoiceemail").textFont = font.TimesB;
   this.getField("clntinvoiceemail").userName = "Enter email for invoices";

   this.getField("clntinvoiceemailnext").textFont = font.TimesB;
   this.getField("clntinvoiceemailnext").userName = "Enter email for invoices";
   
   this.getField("clntbillingemail").textFont = font.TimesB;
   this.getField("clntbillingemail").userName = "Enter email for disputes,CDRs";

   this.getField("clntbillingemailnext").textFont = font.TimesB;
   this.getField("clntbillingemailnext").userName = "Enter email for disputes,CDRs";

   this.getField("clntfinanceemail").textFont = font.TimesB;
   this.getField("clntfinanceemail").userName = "Enter email for other financial questions";

   this.getField("clntfinanceemailnext").textFont = font.TimesB;
   this.getField("clntfinanceemailnext").userName = "Enter email for other financial questions";

   this.getField("clntbeneficiarname").textFont = font.TimesI;
   this.getField("clntbeneficiarname").userName = "Enter beneficiar name";

   this.getField("clntbeneficiarnamenext").textFont = font.TimesI;
   this.getField("clntbeneficiarnamenext").userName = "Enter beneficiar name";

   this.getField("clntbeneficiarbank").textFont = font.TimesI;
   this.getField("clntbeneficiarbank").userName = "Enter beneficiar's bank";

   this.getField("clntbeneficiarbanknext").textFont = font.TimesI;
   this.getField("clntbeneficiarbanknext").userName = "Enter beneficiar's bank";

   this.getField("clntbeneficiarbankswift").textFont = font.TimesI;
   this.getField("clntbeneficiarbankswift").userName = "Enter beneficiar's bank S.W.I.F.T. code";

   this.getField("clntbeneficiarbankswiftnext").textFont = font.TimesI;
   this.getField("clntbeneficiarbankswiftnext").userName = "Enter beneficiar's bank S.W.I.F.T. code";

   this.getField("clntbeneficiarbankiban").textFont = font.TimesI;
   this.getField("clntbeneficiarbankiban").userName = "Enter beneficiar's IBAN";

   this.getField("clntbeneficiarbankibannext").textFont = font.TimesI;
   this.getField("clntbeneficiarbankibannext").userName = "Enter beneficiar's IBAN";

   this.getField("clntcorrespondentbank").textFont = font.TimesI;
   this.getField("clntcorrespondentbank").userName = "Enter correspondent bank";

   this.getField("clntcorrespondentbanknext").textFont = font.TimesI;
   this.getField("clntcorrespondentbanknext").userName = "Enter correspondent bank";

   this.getField("clntcorrespondentbankswift").textFont = font.TimesI;
   this.getField("clntcorrespondentbankswift").userName = "Enter correspondent bank S.W.I.F.T. code";

   this.getField("clntcorrespondentbankswiftnext").textFont = font.TimesI;
   this.getField("clntcorrespondentbankswiftnext").userName = "Enter correspondent bank S.W.I.F.T. code";

   this.getField("clntcorrespondentbankaccount").textFont = font.TimesI;
   this.getField("clntcorrespondentbankaccount").userName = "Enter correspondent bank account number";

   this.getField("clntcorrespondentbankaccountnext").textFont = font.TimesI;
   this.getField("clntcorrespondentbankaccountnext").userName = "Enter correspondent bank account number";
   
   this.getField("clntgmt").textFont = font.TimesB;
   this.getField("clntgmt").userName = "Enter GMT offsets";

   this.getField("clntgmtsummer").textFont = font.TimesB;
   this.getField("clntgmtsummer").userName = "Enter GMT offset";

}
\end{insDLJS}
\OpenAction{/S /JavaScript /JS (docOpen();)}

\def\tfnextlinewidth{145mm}
\def\tfnextlinewidthtwo{139mm}
\def\tfnextlinewidththree{167mm}
\newlength{\fieldlength}
\newcommand{\MyTextField}[2]{
  \setlength{\fieldlength}{\linewidth-\widthof{#1}-3mm}
  #1 #2
}
\newcommand{\MyTextFieldTwo}[2]{
  \setlength{\fieldlength}{\linewidth-\widthof{#1}-2mm}
  #1 #2
}

\begin{document}
\begin{Form}
  \selectlanguage{russian}
  \begin{raggedleft}
    \parbox{10cm}{
    \textbf{\large{\ipxpname}}\\
    Bubenec, Ceskoslovenske armady 371/11,\\
    PSC 160 00, Prague 6, Czech Republic\\
    Telephone: \href{callto:+420226020300}{+420 226 020 300}\\
    Fax: \href{callto:+420226020301}{+420 226 020 301}\\
    Web: \href{http://www.ipxp.net}{http://www.ipxp.net}\\
    E-mail: \href{mailto:agreements@ipxp.net}{agreements@ipxp.net}\\
    }\hfill
  \end{raggedleft}
  \vskip 2\baselineskip
  \begin{center}
   \textbf{\Large{ДОГОВОР О ТЕЛЕКОММУНИКАЦИОННЫХ УСЛУГАХ}}\\
    от \underline{\TextField[name=contractdate,charsize=10pt,width=5em,height=1em,borderwidth=0,
    bordercolor={1 1 1},align=0]{}} \No~\underline{\TextField[name=contractnum,charsize=10pt,width=6em,height=1em,borderwidth=0,
    bordercolor={1 1 1},align=0]{}}
  \end{center}

   \textbf{\ipxpname}, компания в лице директора \textit{Владимира Динкевича},
   учрежденная в соответствии с законодательством \textit{Чешской Республики} под регистрационным
   номером \textit{26756820} и имеющая юридический адрес: \textit{\ipxpaddress} (далее \textbf{\flqqКомпания\frqq})
   и
   \\[\medskipamount]
   \MyTextField{}{\underline{\TextField[name=clntname,charsize=10pt,width=\fieldlength,height=1em,borderwidth=0, bordercolor={1 1 1}, default=Company name]{}},}
   \\[\medskipamount]
   компания в лице\\
   \MyTextField{}{\underline{\TextField[name=clntdirname,charsize=10pt,width=\fieldlength,height=1em,borderwidth=0,bordercolor={1 1 1}, default=Director's name]{}},}
   \\[\medskipamount]
   учрежденная в соответствии с законодательством\\
   \MyTextField{}{\underline{\TextField[name=clntcountry,charsize=10pt,width=\fieldlength,height=1em,borderwidth=0, bordercolor={1 1 1}, default=Country]{}}}
   \\[\medskipamount]
   под регистрационным номером\\
   \MyTextField{}{\underline{\TextField[name=clntregnum,charsize=10pt,width=\fieldlength,height=1em,borderwidth=0, bordercolor={1 1 1}, default=Registration number]{}}}
   \\[\medskipamount]
   и имеющая юридический адрес:\\
   \MyTextField{}{\underline{\TextField[name=clntlegaladdr,charsize=10pt,width=\fieldlength,height=1em,borderwidth=0,bordercolor={1 1 1}, default=Legal address]{}}}
   \\[\medskipamount]
   \MyTextField{}{\underline{\TextField[name=clntlegaladdrnext,charsize=10pt,width=\fieldlength,height=1em,borderwidth=0,bordercolor={1 1 1}]{}}}
   \\[\medskipamount]
   (далее \textbf{\flqqКлиент\frqq}), далее вместе именуемые \textbf{\flqqСтороны\frqq}, заключили
   настоящий договор о нижеследующем:

   \section*{Предмет договора}
   Стороны  согласились объединить свои телекоммуникационные системы через публичную сеть Интернет с целью обеспечения
   телекоммуникационных услуг по передаче голоса поверх протокола IP в соответствии со сроками и условиями,
   изложенными в данном договоре.

   \section{Услуги}
    \begin{enumerate}[label=\thesection.\arabic*.]   
    \item Стороны согласились предоставлять друг другу телекоммуникационные услуги (далее \textbf{\flqqУслуги\frqq}),
          в соответствии с которыми каждая из Сторон может совершать звонки на узел другой Стороны на направления,
          указанные этой Стороной в уведомлениях, составляемых и отправляемых в соответствии с условиями, указанными в
          \hyperref[anx:a-ru]{\flqq{}Приложении~A\frqq{}}

    \item Стороны обязуются использовать Услуги  в соответствии со всеми применимыми
          постановлениями, разрешениями, техническими нормами, указаниями и решениями органов власти страны каждой
          Стороны.

    \item Стороны обязуются прилагать разумные усилия, чтобы гарантировать, что они:
          \begin{enumerate}[label=\theenumi\arabic*.]
            \item не будут использовать Услуги в каких-либо незаконных и противоправных целях
                  и не будут позволять делать это другим;
            \item будут выполнять все разумные распоряжения другой Стороны, касающиеся использования Услуг;
            \item в течение всего срока действия настоящего договора будут иметь все необходимые лицензии
                  на оперирование телекоммуникационными системами и оборудованием, которые должны быть
                  задействованы в сети согласно настоящему договору и гарантируют, что использование услуг
                  не приведет к нарушению любых условий телекоммуникационных лицензий, которыми владеет любая из Сторон.
          \end{enumerate}
    \end{enumerate}

    
   \section{Присоединение}
    \begin{enumerate}[label=\thesection.\arabic*.]
     \item Стороны  должны установить и поддерживать такие основные точки включения, которые в разумных
           пределах необходимы для предоставления Услуг.

     \item Стороны  изначально соединяют свои телекоммуникационные системы через публичную сеть Интернет.
          Процедуры, касающиеся заказа и обеспечения необходимых технических мощностей,
          проводятся каждой Стороной независимо. Изменения и дополнения к указанным процедурам
          также вносятся каждой Стороной независимо, с учетом информации о прогнозируемых объемах трафика,
          предоставленной Сторонами друг другу.
      \item Ни одна из Сторон не несёт никаких обязательств по предоставлению Услуг, если объем или структура
           трафика превышают её технические возможности.
   \end{enumerate}

   \section{Срок действия}\label{sec:3}
    \begin{enumerate}[label=\thesection.\arabic*.]
     \item  Настоящий договор заключается сроком на \textbf{1 (один) год} (далее \textbf{\flqqПериод действия\frqq})
              и вступает в силу со дня подписания.
      \item Действие договора может быть прекращено до истечения Периода действия в соответствии с параграфом \ref{sec:8} настоящего договора.
      \item Срок действия настоящего договора автоматически продлевается на каждый последующий год
            до тех пор, пока он не будет расторгнут любой из Сторон в соответствии с условиями параграфа \ref{sec:8} настоящего договора.
    \end{enumerate}
    \section{Порядок взаиморасчётов}
      \begin{enumerate}[label=\thesection.\arabic*.]
       \item  Стороны  должны обменяться оригиналами данного договора. До этого оплата услуг производиться не будет.      
       \item Каждая Сторона  соглашается оплачивать Услуги, оказанные другой Стороной в соответствии с
             настоящим договором, согласно тарифам, указанным в соответствующих уведомлениях (параграф \ref{enu:4.2}).
       \item Тарифы, назначенные каждой из Сторон другой Стороне, указываются в уведомлениях,
             составляемых и отправляемых в соответствии с условиями, указанными в \hyperref[anx:a-ru]{\flqq{}Приложении~A\frqq{}}
             к настоящему договору.\label{enu:4.2}
%        \item Расчётный период для Сторон~--- \textbf{7 (семь) дней}. Каждая из Сторон  в течение \textbf{2 (двух)
%              дней} по окончании расчётного периода направляет другой Стороне счёт. Все суммы в счетах
%              указываются в долларах США.\label{enu:4.3}
%        \item Сторона  обязуется оплачивать другой Стороне  расходы по терминации трафика в течение \textbf{2 (двух)
%              дней} с момента получения соответствующего счёта. Суммы, подлежащие оплате, должны быть рассчитаны в
%              соответствии с уведомлениями, составленными и отправленными в соответствии с условиями,
%              указанными в \hyperref[anx:a-ru]{\flqq{}Приложении~A\frqq{}} к настоящему договору.
%              В случае равенства задолженности Сторон по оплате за расчётный период обязательства
%              Сторон в части взаиморасчётов прекращаются.
%              При неравенстве встречных обязательств после зачёта большее по размеру обязательство подлежит
%              оплате в части, превышающей меньшее обязательство.\label{enu:4.4}             
%       \item Расчётный период для Сторон~--- \textbf{15 (пятнадцать) дней}. Каждая из Сторон  в течение \textbf{5 (пяти)
%             дней} по окончании расчётного периода направляет другой Стороне счёт. Все суммы в счетах
%             указываются в долларах США.\label{enu:4.3}
     \item Расчётный период для Сторон~--- \textbf{1 (один) месяц}. Каждая из Сторон  в течение \textbf{5 (пяти)
           дней} по окончании расчётного периода направляет другой Стороне счёт. Все суммы в счетах
           указываются в долларах США.\label{enu:4.3}
      \item Сторона  обязуется оплачивать другой Стороне  расходы по терминации трафика в течение \textbf{15 (пятнадцати)
            дней} с момента получения соответствующего счёта. Суммы, подлежащие оплате, должны быть рассчитаны в
            соответствии с уведомлениями, составляемыми и отправляемыми в соответствии с условиями,
            указанными в \hyperref[anx:a-ru]{\flqq{}Приложении~A\frqq{}} к настоящему договору.
            В случае равенства задолженности Сторон по оплате за расчётный период обязательства
            Сторон в части взаиморасчётов прекращаются.
            При неравенстве встречных обязательств после зачёта большее по размеру обязательство подлежит
            оплате в части, превышающей меньшее обязательство.\label{enu:4.4}
       \item Банковские комиссионные расходы несёт Сторона, осуществляющая платёж. \label{enu:4.5}
       \item Согласно параграфу \ref{sec:5}, если любая из Сторон не оплатила выставленный ей счёт в соответствии с параграфами
            \ref{enu:4.3} и \ref{enu:4.4} выше, другая Сторона может начислить пеню в размере \textbf{0,5\% (половины процента)}
             от просроченной суммы за каждый день просрочки, кроме случаев, когда счёт или его часть
             корректно оспариваются, но не более \textbf{10\% (десяти процентов)} от суммы задолженности.\label{enu:4.6}
       \item Цены, указанные в данном договоре, не включают налог на добавленную стоимость~(НДС).
       \item Стороны имеют право выставлять счета за оказанные Услуги раньше
                расчётного периода, если одна из Сторон совершила звонки на
                направления, предоставляемые другой Стороной, на сумму свыше \textbf{2000 (двух тысяч) долларов США}
                (далее \textbf{\flqq{}Кредитный лимит\frqq}).
                В случае превышения данного Кредитного Лимита в течение расчётного
                периода, Сторона, принявшая звонки, должна выставить Стороне, совершившей звонки, счёт, который
                должен быть оплачен в течение \textbf{1 (одного) дня}. В противном случае оказание
                Услуг Стороне, превысившей данный Кредитный Лимит, будет приостановлено до
                погашения задолженности.
       \item В случае если сумма, подлежащая уплате за расчётный период, составила менее \textbf{1000 (одной тысячи) долларов США},
             то указанный платёж переносится в счёт оплаты Услуг, оказанных в следующем расчётном периоде,
             или оплачивается за счёт Стороны-получателя. В любом случае данный платёж должен быть произведён не позднее,
             чем через \textbf{2 (два) месяца} с момента окончания расчётного периода, в котором сумма, подлежащая уплате за
             данный расчётный период, составила менее \textbf{1000 (одной тысячи) долларов США}\@. Пеня в данном случае не начисляется.
      \end{enumerate}


    \section{Разногласия}\label{sec:5}
     \begin{enumerate}[label=\thesection.\arabic*.]
      \item Стороны обязуются сотрудничать для быстрого обнаружения разногласий относительно правильности любых биллинговых
            данных, записанных в сети любой из Сторон, или относительно любых других сумм, подлежащих оплате согласно данному
            договору. Несогласная Сторона должна уведомить другую о любых подобных разногласиях в письменной форме и до даты,
            когда счёт подлежит оплате. Если несогласная сторона не уведомила другую о любых разногласиях до указанной даты,
            счёт считается подтвержденным.
            Стороны допускают расхождение в пределах \textbf{1 (одного) процента} (но не более \textbf{100 (ста) долларов США})
            между суммой счёта, выставленного Стороной  за предоставленные Услуги, и данными Стороны, которой
            выставлен счёт. В таком случае счёт подлежит оплате в полном объёме в срок или до истечения срока оплаты.
            Если расхождения между суммой счёта и данными Стороны, которой выставлен счёт, больше \textbf{1 (одного)
            процента} от суммы счёта или больше \textbf{100 (ста) долларов США}, выплата спорной суммы может быть отложена
            до разрешения спора по письменному согласованию Сторон.
      \item Если расхождения по биллинговым данным остаются неразрешенными в течение \textbf{десяти (10) рабочих дней} после даты,
            когда соответствующий счёт подлежал оплате (если только такой период не был продлен по согласованию Сторон),
            любая Сторона  может (предоставив предварительное письменное уведомление другой Стороне о подобном действии) передать
            разногласия на рассмотрение независимому эксперту (далее \textbf{\flqq{}Эксперт\frqq}), утверждённому обеими Сторонами,
            с просьбой к нему/ней выступить в роли независимого Эксперта, и чьё решение, при отсутствии явной ошибки,
            должно быть окончательным и обязательным.
      \item Стороны  должны взаимодействовать в таком рассмотрении и, если необходимо, любые суммы, таким образом обнаруженные и
            подлежащие оплате или переплаченные, касающиеся счёта, по которому возникли разногласия, должны быть выплачены или
            возмещены (включая любые причитающиеся или оплаченные проценты, относящиеся к параграфу \ref{enu:4.5} выше)
            в течение \textbf{десяти (10) рабочих дней} со дня вынесения решения, согласно условиям или ранее произведенным
            расчётам между Сторонами.
      \item Каждая Сторона  будет нести свои собственные расходы, связанные с вынесением решения Экспертом, а также оплачивает половину гонорара
           Эксперту (или такую часть, которую определит Эксперт), согласованного между Сторонами  ранее за назначение Эксперта.
     \end{enumerate}


    \section{Предоставление информации и конфиденциальность}\label{sec:7}
     \begin{enumerate}[label=\thesection.\arabic*.]
      \item Каждая Сторона  обязуется оперативно предоставлять другой Стороне  всю информацию и содействие, которые может в
            разумных пределах запросить другая Сторона для обеспечения возможности выполнять свои обязательства по данному
            договору или для проверки предъявленных по данным условиям сумм.\label{enu:7.2}
      \item Согласно параграфу \ref{enu:7.3}, каждая Сторона обязуется считать конфиденциальными и не разглашать
              условия настоящего договора, а также все данные, сводки, ставки, отчеты, информацию технического,
              коммерческого или любого другого содержания, связанную любым образом с бизнесом или сделками другой
              Стороны, которую другая Сторона может получить в связи с данным договором.
              Также каждая из Сторон обязуется приложить разумные усилия, чтобы руководство, сотрудники,
              профессиональные консультанты и агенты принимали такую информацию как конфиденциальную, не разглашали и не
              использовали такую информацию, кроме как для целей, непосредственно связанных с настоящим договором,
              за исключением случаев, когда на это получено письменное разрешение другой Стороны.
      \item Условия параграфа \ref{enu:7.2} не распространяются на информации, которая: \label{enu:7.3}
            \begin{enumerate}[label=\theenumi\arabic*.]
              \item распространена или стала всеобщим достоянием без разглашения настоящего договора;
              \item получена Стороной от третьей стороны, имеющей право на ее разглашение;
              \item составляется или составлена Стороной самостоятельно (исключая данные, касающиеся звонков, произведённых другой Стороной);
              \item находится во владении Стороны или стала известна Стороне до подписания настоящего договора, поскольку
                    эта Сторона не была связана никакими обязательствами по конфиденциальности относительно такой информации
                    с другой Стороной.
            \end{enumerate}
      \item Следующие разглашения любой из Сторон  не считаются нарушением параграфа \ref{enu:7.2}:
            \begin{enumerate}[label=\theenumi\arabic*.]
              \item разглашение информации, необходимое в соответствии с законом или действующим распоряжением суда правомочной
                    юрисдикции или указом, распоряжением или приказом любого правительственного или другого регулирующего
                    органа или учреждения, которыми предусматривается, что Сторона,  разглашающая информацию, должна уведомить другую
                    Сторону  о подобном приказе или запросе (и, если возможно, до того как информация будет разглашена) и должна
                    требовать соблюдения конфиденциальности такой информации третьей Стороной, которой данная информация
                    разглашается;
              \item разглашение информации аудитору и/или другому профессиональному консультанту Сторон;
                    или если это является частью обычной отчетности или описательной процедуры для компании учредителя,
                    членов компании учредителя или партнёров, в зависимости от обстоятельств, при условии, что Сторона,
                    разглашающая информацию, приложит усилия, чтобы ее аудиторы, профессиональные консультанты, члены
                    компании учредителя и партнёры рассматривали данную информацию как конфиденциальную;
              \item разглашение информации в целях осуществления Стороной своих прав по настоящему договору;
              \item при расторжении данного договора по какой-либо причине Сторона, получающая извещение о расторжении,
                    должна передать Стороне, разглашающей информацию, или, на усмотрение разглашающей Стороны, уничтожить
                    все копии конфиденциальной информации, полученной от другой Стороны. Условие данного параграфа
                    \ref{sec:7} остается в силе после прекращения или окончания Периода действия настоящего договора,
                    в независимости от причины вышеупомянутого.
            \end{enumerate}
     \end{enumerate}


     \section{Прекращение действия договора}\label{sec:8}
      \begin{enumerate}[label=\thesection.\arabic*.]
       \item  Любая из Сторон может по своему усмотрению и без ущемления своих прав и прав другой Стороны либо
              приостановить, либо прекратить действие настоящего договора, предоставив письменное извещение другой Стороне,
              вступающее в силу немедленно, если выполняется любое из нижеперечисленных условий:\label{enu:7.1}
              \begin{enumerate}[label=\theenumi\arabic*.]
                \item другая Сторона  существенно нарушила какое-либо из своих обязательств, включая задержку оплаты
                      любых счетов, и не возместила нарушение в течение \textbf{5 (пяти) дней} после получения письменного
                      уведомления об этом;
                \item лицензия любой из Сторон  на управление или использование любой из Систем Стороны, которая влияет на
                      выполнение обязательств по данному договору, аннулирована или прекратила
                      свое действие по любой причине (своевременно не заменена);
                \item другая Сторона договаривается или заключает компромиссное соглашение с кредиторами; подает заявление в надлежащую
                      судебную инстанцию о защите от кредиторов; против другой Стороны вынесено распоряжение о банкротстве или другой
                      Стороной принято решение о ликвидации; соответствующая судебная инстанция издает приказ о ее ликвидации или
                      роспуске.
              \end{enumerate}
      \item Любая Сторона  может расторгнуть настоящий договор по любой другой причине, не указанной в пункте \ref{enu:7.1},
            предоставив другой Стороне заранее, по крайней мере за \textbf{30 (тридцать) дней} до планируемой даты расторжения
            договора, письменное уведомление о своём нежелании продлевать действие договора.
      \item Любая Сторона  может расторгнуть настоящий договор при возникновении событий, указанных в параграфе \ref{sec:10}.
      \item Расторжение или прекращение Периода действия настоящего договора не влияет на приобретенные Сторонами права
            на момент расторжения или прекращения срока действия или на обязательства продолжительного характера,
            которые должны выполняться, включая, без ограничений, обеспечение возмещения, конфиденциальности и обязательства
            произвести платежи, несмотря на расторжение или прекращение срока действия настоящего договора по любым причинам.
      \end{enumerate}

    \section{Ограничение ответственности}\label{sec:9}
        \begin{enumerate}[label=\thesection.\arabic*.]
         \item Никакая из Сторон  не должна ни при каких обстоятельствах нести ответственность за любые косвенные,
               побочные, специальные или случайные убытки или ущерб, причиненные другой Стороне, ее служащим или агентам,
               которые могут возникнуть в связи с настоящим договором (включая, без ограничений, потерю репутации, бизнеса
               или прибыли).
         \item Условия настоящего параграфа остаются в силе, несмотря на прекращение или окончание срока действия
               настоящего договора по любой какой бы то ни было причине.
        \end{enumerate}

    \section{Обстоятельства непреодолимой силы}\label{sec:10}
        \begin{enumerate}[label=\thesection.\arabic*.]
         \item  Несмотря на любые условия настоящего договора, ни одна из Сторон  не должна нести ответственность за свою
          неспособность в исполнении любых своих обязательств (не считая обязательств по платежам), если такая неспособность
          вызвана или возникает в результате обстоятельств, выходящих за рамки разумного контроля соответствующей Стороны включая,
          без ограничений, неспособность или задержку, вызванную следующими событиями: стихийное бедствие, пожар, наводнение, восстание,
          конфликты любого рода в промышленности (не считая конфликтов, в которые включены собственные работники Стороны  или
          работники ассоциированной компании данной Стороны; в рамках данного параграфа ассоциированная компания понимается
          как отделение, филиал или контролируемая организация), молнии, взрывы, общественные беспорядки, ущерб, совершенный со злым
          умыслом, шторм, буря, действия или ошибки других коммуникационных операторов, действия правительства или других
          регулирующих органов, действия или ошибки физических или юридических лиц, за которые сторона не несет ответственности, и
          любые другие обстоятельства, выходящие за рамки разумного контроля соответствующей Стороны (далее \textbf{\flqq{}Форс-мажор\frqq{}}).
          \item Сторона, попавшая под влияние форс-мажорных обстоятельств, должна своевременно уведомить другую Сторону о предполагаемой
           степени и длительности подобной неспособности выполнения обязательств по договору. и в случае, если настоящий
           договор не может быть выполнен в соответствии с условиями в течение \textbf{30 (тридцати) дней} по причине форс-
           мажорных обстоятельств, Сторона  имеет право официально известить другую Сторону о прекращении настоящего
           договора без каких-либо обязательств по отношению к этой Стороне; условия продолжительного характера должны
           продолжать выполняться, несмотря на прекращение действия договора по данному пункту.
           
        \end{enumerate}

    \section{Гарантии}\label{sec:11}
        \begin{enumerate}[label=\thesection.\arabic*.]
         \item Каждая Сторона  обязуется прилагать разумные усилия  для поддержания качества всей своей сети;
         \item Качество Услуги, предоставляемой согласно настоящему договору, должно строго соответствовать другим общим
               стандартам операторской отрасли, правительственным постановлениям и разумной практике деловых отношений. Никакие
               другие гарантии не могут быть даны любой из Сторон  другой Стороне  или любым другим лицам или организациям
               касательно предоставляемых Услуг, включая, но без ограничений, любые гарантии работоспособности состояния или
               пригодности для данных целей.
        \end{enumerate}

    \section{Переуступка прав}
        \begin{enumerate}[label=\thesection.\arabic*.]
         \item Компания  не может передать, уступить или претендовать на передачу или уступку любых своих прав или обязательств
               по настоящему договору без предварительного получения письменного согласия другой Стороны. Несмотря на
               вышеупомянутое, любая Сторона  может передать в письменной форме свои права и обязательства по настоящему
               договору ассоциированной компании этой Стороны (в рамках данного параграфа ассоциированная компания понимается
               как отделение, филиал или контролируемая организация) или третьей Стороне, которая приобрела её систему,
               если такая ассоциированная компания или третья Сторона  имеет все необходимые
               лицензии, разрешения и согласования, которые могут быть необходимы для выполнения обязательств как правопреемной по
               настоящему договору.
         \item Любое подобное соглашение, разрешённое здесь, вступает в силу только после подписания его
               обеими Сторонами  и правопреемником формального соглашения по передаче прав, по которому правопреемник должен
               согласиться выполнять все условия настоящего договора, относящиеся к Стороне, выступающей преемником.
        \end{enumerate}
        

    \section{Права интеллектуальной собственности}
        \begin{enumerate}[label=\thesection.\arabic*.]
         \item Кроме случаев, которые специально оговорены в письменном виде между Сторонами  (взаимно согласованные условия), все
               торговые и сервисные марки, изобретения, патенты, копирайты, зарегистрированные дизайны, права дизайна и все другие
               права интеллектуальной собственности должны быть и оставаться в собственности соответствующей Стороны.
        \end{enumerate}

    \section{Уведомления}%FIXME форматирование полей
        \begin{enumerate}[label=\thesection.\arabic*.]
         \item Все уведомления, разрешения, отказы или другие сообщения должны выполняться в письменном форме и доставляться курьером,
               заказной или сертифицированной почтой (расписка о получении необходима) или отправляться по факсу или электронной
               почте и считаются полученными при действительной доставке. Все уведомления должны направляться на
               соответствующие адреса: % FIXME перевод
               \begin{itemize}
                \item Для компании \textbf{\ipxpname}:
                      \begin{itemize}
                        \item Адрес: \textit{\ipxpaddress}
                        \item Телефон: \textit{+420 226 020 300}
                        \item Факс: \textit{+420 226 020 301}
                        \item E-mail:
                            \begin{itemize}
                              \item \MyTextField{Контактное лицо:}
                                {\underline{\TextField[name=ipxpmanager,charsize=10pt,width=\fieldlength,height=1em,borderwidth=0,bordercolor={1 1 1}]{}}}
                              \item Общие вопросы поддержки: \href{mailto:support@ipxp.net}{\textbf{support@ipxp.net}}
                              \item Технические вопросы: \href{mailto:noc@ipxp.net}{\textbf{noc@ipxp.net}}
                              \item Вопросы качества предоставляемых услуг: \href{mailto:qa@ipxp.net}{\textbf{qa@ipxp.net}}
                              \item Уведомления о предлагаемых направлениях и об изменении тарифов:
                                    \href{mailto:rates@ipxp.net}{\textbf{rates@ipxp.net}}
                              \item Договоры, дополнительные соглашения: \href{mailto:agreements@ipxp.net}{\textbf{agreements@ipxp.net}}
                              \item Инвойсы: \href{mailto:invoices@ipxp.net}{\textbf{invoices@ipxp.net}}
                              \item Диспуты, CDR: \href{mailto:billing@ipxp.net}{\textbf{billing@ipxp.net}}
                              \item Другие финансовые вопросы: \href{mailto:finance@ipxp.net}{\textbf{finance@ipxp.net}} % тут спрашивать надо об облаченных счетах и т.п.
                            \end{itemize}
                      \end{itemize}
                \item \MyTextField{Для компании }{\underline{\TextField[name=clntname,charsize=10pt,width=\fieldlength,height=1em,borderwidth=0,bordercolor={1 1 1}]{}}:}
                      \begin{itemize}
                        \item \MyTextField{Адрес:}
                                {\underline{\TextField[name=clntcorraddr,charsize=10pt,width=\fieldlength,height=1em,borderwidth=0,bordercolor={1 1 1}]{}}}
                                \\[\medskipamount]
                                \MyTextFieldTwo{}{\underline{\TextField[name=clntcorraddrnext,charsize=10pt,width=\fieldlength,height=1em,borderwidth=0,bordercolor={1 1 1}]{}}}
                        \item \MyTextField{Телефон:}
                                {\underline{\TextField[name=clntphone,charsize=10pt,width=\fieldlength,height=1em,borderwidth=0,bordercolor={1 1 1}]{}}}
                                \\[\medskipamount]
                                \MyTextFieldTwo{}{\underline{\TextField[name=clntphonenext,charsize=10pt,width=\fieldlength,height=1em,borderwidth=0,bordercolor={1 1 1}]{}}}
                        \item \MyTextField{Факс:}
                                {\underline{\TextField[name=clntfax,charsize=10pt,width=\fieldlength,height=1em,borderwidth=0,bordercolor={1 1 1}]{}}}
                                \\[\medskipamount]
                                \MyTextFieldTwo{}{\underline{\TextField[name=clntfaxnext,charsize=10pt,width=\fieldlength,height=1em,borderwidth=0,bordercolor={1 1 1}]{}}}
                        \item E-mail:
                            \begin{itemize}
                              \item \MyTextField{Контактное лицо:}
                                {\underline{\TextField[name=clntmanager,charsize=10pt,width=\fieldlength,height=1em,borderwidth=0,bordercolor={1 1 1}]{}}}
                                \\[\medskipamount]
                                \MyTextFieldTwo{}{\underline{\TextField[name=clntmanagernext,charsize=10pt,width=\fieldlength,height=1em,borderwidth=0,bordercolor={1 1 1}]{}}}
                              \item \MyTextField{Общие вопросы поддержки:}
                                    {\underline{\TextField[name=clntgeneralemail,charsize=10pt,width=\fieldlength,height=1em,borderwidth=0,bordercolor={1 1 1}]{}}}
                                    \\[\medskipamount]
                                    \MyTextFieldTwo{}{\underline{\TextField[name=clntgeneralemailnext,charsize=10pt,width=\fieldlength,height=1em,borderwidth=0,bordercolor={1 1 1}]{}}}
                              \item \MyTextField{Технические вопросы:}
                                    {\underline{\TextField[name=clntnocemail,charsize=10pt,width=\fieldlength,height=1em,borderwidth=0,bordercolor={1 1 1}]{}}}
                                    \\[\medskipamount]
                                    \MyTextFieldTwo{}{\underline{\TextField[name=clntnocemailnext,charsize=10pt,width=\fieldlength,height=1em,borderwidth=0,bordercolor={1 1 1}]{}}}
                              \item \MyTextField{Вопросы качества предоставляемых услуг:}
                                    {\underline{\TextField[name=clntqaemail,charsize=10pt,width=\fieldlength,height=1em,borderwidth=0,bordercolor={1 1 1}]{}}}
                                    \\[\medskipamount]
                                    \MyTextFieldTwo{}{\underline{\TextField[name=clntqaemailnext,charsize=10pt,width=\fieldlength,height=1em,borderwidth=0,bordercolor={1 1 1}]{}}}
                              \item \MyTextField{Уведомления о предлагаемых направлениях и об изменении тарифов:}
                                    {\underline{\TextField[name=clntratesemail,charsize=10pt,width=\fieldlength,height=1em,borderwidth=0,bordercolor={1 1 1}]{}}}
                                    \\[\medskipamount]
                                    \MyTextFieldTwo{}{\underline{\TextField[name=clntratesemailnext,charsize=10pt,width=\fieldlength,height=1em,borderwidth=0,bordercolor={1 1 1}]{}}}
                              \item \MyTextField{Договоры, дополнительные соглашения:}
                                    {\underline{\TextField[name=clntagreementsemail,charsize=10pt,width=\fieldlength,height=1em,borderwidth=0,bordercolor={1 1 1}]{}}}
                                    \\[\medskipamount]
                                    \MyTextFieldTwo{}{\underline{\TextField[name=clntagreementsemailnext,charsize=10pt,width=\fieldlength,height=1em,borderwidth=0,bordercolor={1 1 1}]{}}}
                              \item \MyTextField{Инвойсы:}
                                    {\underline{\TextField[name=clntinvoiceemail,charsize=10pt,width=\fieldlength,height=1em,borderwidth=0,bordercolor={1 1 1}]{}}}
                                    \\[\medskipamount]
                                    \MyTextFieldTwo{}{\underline{\TextField[name=clntinvoiceemailnext,charsize=10pt,width=\fieldlength,height=1em,borderwidth=0,bordercolor={1 1 1}]{}}}
                              \item \MyTextField{Диспуты, CDR:}
                                    {\underline{\TextField[name=clntbillingemail,charsize=10pt,width=\fieldlength,height=1em,borderwidth=0,bordercolor={1 1 1}]{}}}
                                    \\[\medskipamount]
                                    \MyTextFieldTwo{}{\underline{\TextField[name=clntbillingemailnext,charsize=10pt,width=\fieldlength,height=1em,borderwidth=0,bordercolor={1 1 1}]{}}}
                              \item \MyTextField{Другие финансовые вопросы:}
                                    {\underline{\TextField[name=clntfinanceemail,charsize=10pt,width=\fieldlength,height=1em,borderwidth=0,bordercolor={1 1 1}]{}}}
                                    \\[\medskipamount]
                                    \MyTextFieldTwo{}{\underline{\TextField[name=clntfinanceemailnext,charsize=10pt,width=\fieldlength,height=1em,borderwidth=0,bordercolor={1 1 1}]{}}}
                            \end{itemize}
                      \end{itemize}
                      

               \end{itemize}
               Любая Сторона  может изменить свой адрес, номер телефона, факса, адрес электронной почты,
               отправив уведомление об этом, как указано выше. %FIXME перевод
         \item Сообщения, доставляемые курьерской службой, заказной или сертифицированной почтой (расписка о получении необходима)
               или электронной почтой, считаются доставленными в момент получения.
         \item Сообщения, доставляемые по факсимильной связи, считаются полученными при передаче, с условием, что отправитель
               имеет полученный отчёт о передаче, в котором указывается, что все страницы сообщения были переданы на правильный
               номер факса.
         \item Если передача такого факса не произошла в обычный рабочий день и в обычные рабочие часы,
               сообщение считатается полученным на следующий рабочий день. В связи с этим, рабочими считаются любые дни
               за исключением субботы, воскресенья или официальных праздничных дней; под рабочими часами должно подразумевается время
               с 10:00 до 18:00 в рабочий день с учётом часового пояса получателя.
                       
        \end{enumerate}


    \section{Применимое законодательство}\label{sec:law}
        \begin{enumerate}[label=\thesection.\arabic*.]
         \item Настоящий договор составлен в соответствии и руководствуясь во всех отношениях
               законодательством \textbf{Чешской республики}.
         \item Стороны безоговорочно подчиняются юрисдикции суда вышеуказанной страны в отношении каких-либо исков
               или разбирательств, возникающих в рамках настоящего договора.
        \end{enumerate}%FIXME перевод

    \section{Разное}
        \begin{enumerate}[label=\thesection.\arabic*.]
         \item Настоящий договор со всеми его дополнениями и приложениями представляет собой единое соглашение,
               достигнутое между Сторонами касательно предмета договора, и отменяет все другие договоры и соглашения,
               сделанные любой Стороной устно или письменно.
        \end{enumerate}

    \section{Банковские реквизиты}%FIXME форматирование полей
        \begin{enumerate}[label=\thesection.\arabic*.]
          \item  Оплата должна производиться в долларах США банковским переводом. Банковская комиссия за перевод взимается со Стороны,
                оплачивающей счёт.
                \begin{itemize}
                 \item Для компании \textbf{\ipxpname}:
                    \begin{itemize}
                      \item \textbf{Получатель}: \textit{\ipxpname, \ipxpaddress}
                      \bigskip
                    \end{itemize}
                    \begin{itemize}
                      \item \textbf{Банк получателя}: \textit{RAIFFEISENBANK A.S., Na Prikope 19117 19 Prague 1, Czech Republic}
                      \item \textbf{S.W.I.F.T. код}: \textit{RZBCCZPPXXX}
                      \item \textbf{IBAN}:  \textit{CZ0524000000002212430001}
                    \end{itemize}
                    %\smallskip
                    \begin{itemize}
                      \item \textbf{Банк-корреспондент}: \textit{THE BANK OF NEW YORK, New York, USA}
                      \item \textbf{S.W.I.F.T. код}: \textit{IRVT US 3N}
                      \item \textbf{Корреспондентский счёт USD}: \textit{8900492619}
                      \bigskip
                    \end{itemize}
                    \begin{itemize}
                      \item \textbf{Банк получателя}: \textit{JSC \flqq{}REGIONAL INVESTMENT BANK\frqq{}, 2, J. Alunana street, Riga, Latvia}
                      \item \textbf{S.W.I.F.T. код}: \textit{RIBRLV22}
                      \item \textbf{IBAN}:   \textit{LV58RIBR00028530N0000}
                    \end{itemize}
                    %\smallskip
                    \begin{itemize}
                      \item \textbf{Банк-корреспондент}: \textit{Raiffeisen Zentralbank, Vienna, Austria}
                      \item \textbf{S.W.I.F.T. код}: \textit{RZBAATWW}
                      \item \textbf{Корреспондентский счёт USD}: \textit{70-55.065.965}
                    \end{itemize}
                \item \MyTextField{Для компании }{\underline{\TextField[name=clntname,charsize=10pt,width=\fieldlength,height=1em,borderwidth=0,bordercolor={1 1 1}]{}}:}
                    \begin{itemize}
                      \item \MyTextField{\textbf{Получатель}:}
                         {\underline{\TextField[name=clntbeneficiarname,charsize=10pt,width=\fieldlength,height=1em,borderwidth=0,bordercolor={1 1 1}]{}}}
                         \\[\medskipamount]
                         \MyTextFieldTwo{}{\underline{\TextField[name=clntbeneficiarnamenext,charsize=10pt,width=\fieldlength,height=1em,borderwidth=0,bordercolor={1 1 1}]{}}}
                      \bigskip
                    \end{itemize}
                    \begin{itemize}
                      \item \MyTextField{\textbf{Банк получателя}:}
                              {\underline{\TextField[name=clntbeneficiarbank,charsize=10pt,width=\fieldlength,height=1em,borderwidth=0,bordercolor={1 1 1}]{}}}
                              \\[\medskipamount]
                              \MyTextFieldTwo{}{\underline{\TextField[name=clntbeneficiarbanknext,charsize=10pt,width=\fieldlength,height=1em,borderwidth=0,bordercolor={1 1 1}]{}}}
                      \item \MyTextField{\textbf{S.W.I.F.T. код}:}
                               {\underline{\TextField[name=clntbeneficiarbankswift,charsize=10pt,width=\fieldlength,height=1em,borderwidth=0,bordercolor={1 1 1}]{}}}
                               \\[\medskipamount]
                               \MyTextFieldTwo{}{\underline{\TextField[name=clntbeneficiarbankswiftnext,charsize=10pt,width=\fieldlength,height=1em,borderwidth=0,bordercolor={1 1 1}]{}}}
                      \item \MyTextField{\textbf{IBAN}:}
                              {\underline{\TextField[name=clntbeneficiarbankiban,charsize=10pt,width=\fieldlength,height=1em,borderwidth=0,bordercolor={1 1 1}]{}}}
                              \\[\medskipamount]
                               \MyTextFieldTwo{}{\underline{\TextField[name=clntbeneficiarbankibannext,charsize=10pt,width=\fieldlength,height=1em,borderwidth=0,bordercolor={1 1 1}]{}}}
                    \end{itemize}
                    %\smallskip
                    \begin{itemize}
                      \item \MyTextField{\textbf{Банк-корреспондент}:}
                              {\underline{\TextField[name=clntcorrespondentbank,charsize=10pt,width=\fieldlength,height=1em,borderwidth=0,bordercolor={1 1 1}]{}}}
                              \\[\medskipamount]
                              \MyTextFieldTwo{}{\underline{\TextField[name=clntcorrespondentbanknext,charsize=10pt,width=\fieldlength,height=1em,borderwidth=0,bordercolor={1 1 1}]{}}}
                      \item \MyTextField{\textbf{S.W.I.F.T. код}:}
                              {\underline{\TextField[name=clntcorrespondentbankswift,charsize=10pt,width=\fieldlength,height=1em,borderwidth=0,bordercolor={1 1 1}]{}}}
                              \\[\medskipamount]
                              \MyTextFieldTwo{}{\underline{\TextField[name=clntcorrespondentbankswiftnext,charsize=10pt,width=\fieldlength,height=1em,borderwidth=0,bordercolor={1 1 1}]{}}}
                      \item \MyTextField{\textbf{Корреспондентский счёт}:}
                              {\underline{\TextField[name=clntcorrespondentbankaccount,charsize=10pt,width=\fieldlength,height=1em,borderwidth=0,bordercolor={1 1 1}]{}}}
                              \\[\medskipamount]
                              \MyTextFieldTwo{}{\underline{\TextField[name=clntcorrespondentbankaccountnext,charsize=10pt,width=\fieldlength,height=1em,borderwidth=0,bordercolor={1 1 1}]{}}}
                    \end{itemize}
                \end{itemize}

                
        \end{enumerate}

\thispagestyle{agreement-nofooter}

\noindent
\dotfill
\noindent
\vfill
\noindent
\parbox[t]{0.5\linewidth}{
Компания:\hfill\textbf{\ipxpname}\hspace*{0.5cm}\\
ФИО:\hfill\textit{Volodymyr Dinkevych}\hspace*{0.5cm}\\
Должность:\hfill\textit{Директор}\hspace*{0.5cm}\\
Подпись: \hrulefill\hspace*{0.5cm}\\
Дата: \hrulefill\hspace*{0.5cm}\\
}
\hfill
\parbox[t]{0.5\linewidth}{
\hspace*{0.5cm}Компания: \TextField[name=clntname,charsize=10pt,width=62mm,height=1em,align=2,borderwidth=0,bordercolor={1 1 1}]{}\\
\hspace*{0.5cm}ФИО: \TextField[bordersep=1,name=clntattorneyprintname,charsize=10pt,width=69mm,height=1em,align=2,borderwidth=0,bordercolor={1 1 1}]{}\\
\hspace*{0.5cm}Должность: \TextField[name=clntattorneyposition,charsize=10pt,width=60mm,height=1em,align=2,borderwidth=0,bordercolor={1 1 1}]{}\\
\hspace*{0.5cm}Подпись: \hrulefill\\
\hspace*{0.5cm}Дата: \hrulefill\\
}


% begin Appendix A
% Reset section counter and page numbering because it's an appendix
\newpage
\appendix
\setcounter{page}{1}
\pagestyle{appendix}
\thispagestyle{appendix-noheader}

 \selectlanguage{russian}
  
  \begin{raggedleft}
    \hfill\parbox{5cm}{
    \textbf{ПРИЛОЖЕНИЕ~А}\\
    к договору о \\
    телекоммуникационных услугах\\
    от \underline{\TextField[name=contractdate,charsize=10pt,width=5em,height=1em,borderwidth=0,
    bordercolor={1 1 1},align=0]{}}~\No~\underline{\TextField[name=contractnum,charsize=10pt,width=6em,
    height=1em,borderwidth=0,bordercolor={1 1 1},align=0]{}}
    }
  \end{raggedleft}

  \vskip 2\baselineskip

  \begin{center}
    \textbf{\Large{Изменения цен и уведомления}}
  \end{center}
  
  
  \par\textbf{Компания"=поставщик}~--- компания, принимающая трафик. \textbf{Компания"=клиент}~---
  компания, направляющая трафик. Далее вместе именуются \textbf{Стороны}.

 \section{Общие принципы}\label{anx:a-ru}

     \begin{enumerate}[label=\thesection.\arabic*.]
      \item Стороны достигли договорённости о том, что ценообразование является исключительным правом
        каждой Стороны и может не совпадать с ценовой политикой рынка. 
      \item Любая из Сторон может выступить с инициативой об изменении цен или
        кодов, направив другой Стороне письменное уведомление (далее \textbf{\flqq{}Уведомление\frqq{}}).
      \item В случае изменения цен на оказываемые Сторонами услуги, а также изменения
        перечня услуг Стороны обязуются направлять уведомления об этом на
        следующие электронные адреса: 
      
        \begin{itemize}
          \item для компании \textbf{\ipxpname}: \href{mailto:rates@ipxp.net}{\textbf{rates@ipxp.net}}
          \item для \underline{\TextField[name=clntname,charsize=10pt,width=19em,height=1em,borderwidth=0,bordercolor={1 1 1}]{}}:
              \underline{\TextField[name=clntratesemail,charsize=10pt,width=75mm,height=1em,borderwidth=0,
              bordercolor={1 1 1}]{}}
              \\[\medskipamount]
              \underline{\TextField[name=clntratesemailnext,charsize=10pt,width=151mm,height=1em,borderwidth=0,
              bordercolor={1 1 1}]{}}
        \end{itemize}

      \end{enumerate}
      
\section{Содержание уведомлений}
   
    \begin{enumerate}[label=\thesection.\arabic*.]
    \item В случае какого"=либо изменения цены на любое направление Компания"=поставщик
      должна чётко указать в уведомлении изменения по кодам. Каждое уведомление
      обязано содержать: 

      \begin{itemize}
      \item Код;
      \item Название направления;
      \item Стоимость минуты;
      \item Тарификация (1/1~--- посекундная, 60/60~--- поминутная, или др.)~---
        для каждого кода, либо единая для всех кодов в данном уведомлении; 
      \item Статус (повышение цены; понижение цены; удаление кода или подкода
        из прайс"=листа, прописанного в предыдущих уведомлениях; введение нового
        кода или подкода в прайс"=лист; цена без изменения; блокирование);
      \item Дату вступления в силу~--- для каждого кода либо единую для всех
        кодов в данном уведомлении;
      \item Статус ранее предоставляемых подкодов на данный код (если они подлежат
        изменению).
      \end{itemize}

    Каждый код должен занимать отдельную строку таблицы во вложенном файле
    формата \selectlanguage{english}{Comma Separated Values (CSV), Microsoft Excel (XLS)} \selectlanguage{russian}{или}
    \selectlanguage{english}{OpenDocument Spreadsheet (ODS)}. \selectlanguage{russian}Формат CSV явлется более предпочтительным,
    так как остальные форматы имеют ограничение на количество строк (65536) и колонок (256) на одном листе.

  \item \label{enu:2.2-ru}Цена конкретного направления соотносится только с соответствующим
    ему кодом. Название направления указывается исключительно с информативной
    целью.
  \item Маршрутизация трафика происходит на все подкоды и коды, предоставляемые
    поставщиком. При этом трафик маршрутизируется на самый длинный подкод,
    предоставленный поставщиком на это направление. \textit{Например, если цена
    на код 234806 составляет 0,13~USD, а на код 23480~---
    0,09~USD, то звонок на номер 234806121212 протарифицируется
    по цене 0,13~USD. На все остальные коды 23480{*}~--- по цене
    0,09~USD}.
  \item Маршрутизация на более длинные подкоды, начинающиеся с цифры основного
    кода, но не выделенные в прайс"=листе, будет проходить на основной
    код. \textit{Например, если поставщик предоставляет код 1, но не
    указывает в прайс"=листе отдельно код на Доминиканскую республику (1809),
    то звонок на 1809121212 будет тарифицироваться по цене, указанной
    для основного кода 1}. Такие более длинные коды должны быть
    либо заблокированы для приема трафика, либо прописаны отдельным списком
    с указанием цены или статуса \textbf{\flqq{}block\frqq{}}.
  \item Если Компания"=поставщик предоставляет новый код стационарной телефонной
    сети страны (далее \textbf{ \flqq{}PSTN\frqq{}}) (\textit{например, 380}),
    она \textbf{обязана} в этом уведомлении указать статус мобильных кодов и городов,
    если они имеют цену, отличную от цены PSTN (выше или ниже). Если же
    цены на коды мобильных операторов (\textit{38067, 38050
    и т.д.}) не указаны в прайс"=листе, то весь трафик на них будет тарифицироваться
    по цене PSTN, так как иное не предусмотрено уведомлениями от Компании"=поставщика.
  \item Если Сторона при предоставлении кода не может обеспечить работу каких"=либо
    подкодов, такие подкоды должны быть указаны как заблокированные. Компания"=клиент
    должна обеспечить их блокировку со своей стороны и не направлять трафик
    на данные подкоды.
  \item Трафик на направления, не указанные в прайс листе, будет тарифицироваться
    по цене \textbf{10 (десять) долларов США} за минуту.
  \item Во избежание случаев разного трактования уведомления, Компания"=поставщик
    должна при изменении цены на отдельные подкоды отображать в тексте
    уведомления все ранее открытые коды для данной страны с указанием
    их статусов (повышение, понижение, без изменения). В случае отсутствия
    детализации по подкодам они остаются без изменения, сохраняя свою
    прежнюю цену, согласно более ранним уведомлениям. 
  \item Более короткие (основные) коды, вне зависимости от того, их цена выше
    или ниже, чем у подкодов, \textbf{не замещают} действие подкодов, которые продолжают
    действовать в случае отсутствия особых указаний от Компании"=поставщика.
    \textit{Например, если Компания"=поставщик присылает код 79 по цене
    0,06~USD, но ранее предоставляла подкод 7903 по цене 0,04~USD,
    то трафик на номер 7903797979 будет протарифицирован
    по 0,04~USD, так как не было уведомления от поставщика об удалении
    этого подкода, повышении цены на него или его включении в основной
    код}. 
  \item Если Компания"=поставщик подразумевает, что более короткий код замещает
    действие всех более длинных подкодов, предоставляемых ранее, она \textbf{обязана}
    подробно описать это в уведомлении во избежание двойного трактования:
    \begin{itemize}
    \item либо предоставив полный список подкодов, подлежащих удалению, с указанием
      статуса \textbf{\flqq{}de\-le\-te\frqq{}};
    \item либо указав в теле уведомления,
      что с момента вступления уведомления в силу все подкоды тарифицируются
      по цене основного кода (см.\ пункт \ref{enu:2.11-ru})
    \end{itemize}
    В случае отсутствия пояснения о статусе подкодов цена на основной код
    прописывается, а подкоды не удаляются и тарифицируются по ранее указанным ценам. 
  \item \label{enu:2.11-ru}В случае полной замены прайс"=листа Компанией"=поставщиком
    либо полной замены тарифов на определенное направление в \textbf{теле письма}
    обязательно должно быть указано, что данное уведомление \textbf{полностью
    заменяет} предыдущие по всем направлениям либо по определенному направлению,
    обозначенному кодом (\textit{например, Узбекистан 998}). Пример такого
    уведомления: 
    \begin{quote}
      \begin{center}
        \texttt{Уважаемые Коллеги!}
      \end{center}
      \begin{flushleft}
        \indent\texttt{Официально уведомляем Вас об изменении тарифов и кодов на терминацию
        трафика с 1 сентября 2009~г.\\
        \indentПросим обратить Ваше внимание, что терминация трафика будет производиться
        \textbf{только по приведенным в данном уведомлении} кодам и тарифам по направлению
        Uzbekistan (998).\\
        \indentРанее действующие тарифы и коды на терминацию трафика по этому направлению
        просим \textbf{считать недействительными}}.\\
    \end{flushleft}
  \end{quote}
    Уведомление о полной замене тарифов по какому"=либо направлению без
    указания кода этого направления, а только лишь по названию, не может
    считаться действительным в соответствии с пунктом \ref{enu:2.2-ru} настоящего
    Приложения. 
    
  \item Если обслуживание какого"=либо подкода приостанавливается и он подлежит
    удалению из прайс"=листа, Компания"=поставщик может поступить двумя
    способами: 
    
    \begin{itemize}
    \item прислать уведомление о повышении цены на данный подкод до уровня цены
      основного кода; 
    \item или прислать уведомление об удалении данного подкода и его включении
      в основной код (а также повышении цены до уровня основного кода). 
    \end{itemize}

  \item Если Компания"=поставщик присылает коды со статусом \textbf{\flqq{}без
      изменений\frqq{}}, то цены на эти коды не изменяются в системе Компании"=клиента.
    Поэтому Компания"=поставщик несет ответственность за то, чтобы цены
    на эти коды соответствовали ранее присланным.
  \item Компания"=поставщик согласна с тем, что Компания"=клиент \textbf{не несет} ответственности
    за ошибки в уведомлениях или биллинге Компании"=поставщика. Если Компания"=поставщик
    прислала неверные цены и до даты вступления в силу этих цен не прислала
    уточнения, цены считаются принятыми и верными и оспариванию или
    пересчёту не подлежат. \textit{Например, если поставщик прислал цену на код
    1 равную 0,01~USD и не указал в прайс"=листе отдельно цены
    на подкоды 1809,1767 (Доминиканская республика), но принял трафик
    на эти подкоды, он не может требовать пересчёта цены, потому что он
    не выделил эти подкоды как самостоятельные в прайс"=листе}.
  \item Компания"=клиент согласна с тем, что Компания"=поставщик \textbf{не несет} ответственности
    за неверное внесение кодов и тарифов в биллинговую систему Компании"=клиента.
  \item В случае если присланное уведомление дает возможность для двоякого
    трактования и злоупотребления ошибкой Компании"=поставщика, Компания"=клиент
    должна обратиться к Компании"=поставщику для \textbf{уточнения и получения
    дополнительных инструкций}.
  \end{enumerate}
  

\section{Сроки вступления в силу изменений}

  \begin{enumerate}[label=\thesection.\arabic*.]
    \item Стороны договорились, что уведомления о тарифах будут вступать в силу, а расчёты биллинга производиться в следующих временных зонах:
      \begin{itemize}
        \item от \textbf{\ipxpname}: \textbf{GMT+0} (в летнее время: \textbf{GMT+1})
        \item \MyTextField{от}
                {\underline{\TextField[name=clntname,charsize=10pt,width=22em,height=1em,borderwidth=0, bordercolor={1 1 1}, default=Company name]{}}:}
                 \textbf{GMT}\underline{\TextField[name=clntgmt,charsize=10pt,width=3em,maxlen=3,height=1em,borderwidth=0,bordercolor={1 1 1}]{}}
         (в летнее время: \textbf{GMT}\underline{\TextField[name=clntgmtsummer,charsize=10pt,width=3em,maxlen=3,height=1em,borderwidth=0,bordercolor={1 1 1}]{}})
      \end{itemize}
    \item Повышения вступают в силу не ранее, чем через \textbf{7 (семь) дней} с момента уведомления Компании"=клиента.
    \item Понижения вступают в силу с момента уведомления Компании"=клиента.
    \item Удаление кодов вступает в силу не ранее, чем через \textbf{7 (семь) дней} с момента уведомления Компании"=клиента.
    \item Новые коды, если они влекут за собой повышение (\textit{например, из 79 Russia mobile по цене
          0,050~USD выделен новый код 7954 по цене 0,580~USD}), вступают в силу
          не ранее, чем через \textbf{7 (семь) дней} с момента уведомления Компании"=клиента.
          В случае если новые коды не влекут за собой повышение, они вступают в действие с момента уведомления Компании"=клиента.
    \item Полная замена прайс"=листа либо полная замена всех тарифов на определенное
          направление вступает в силу не ранее, чем через \textbf{7 (семь) дней} с момента уведомления
          Компании"=клиента. Уведомление, помеченное как полная замена прайс листа либо полная
          замена всех тарифов на определенное направление, должно иметь единую дату вступления в действие.
    \item Если Компания"=поставщик присылает изменения, вступающие в силу через \textbf{7 (семь) дней},
          то при изменении цен на эти же коды повторно в течение этих же \textbf{7 (семи) дней}
          Компания"=поставщик обязана дополнительно указать, которая из цен останется действительной по истечении
          \textbf{7 (семи) дней}. \textit{Например, если Компания"=поставщик прислала повышение на код 9989 с 0,060
          до 0,063~USD, которое вступит в силу с 21 апреля, а 19 апреля прислала понижение на код 9989 с 0,060 до
          0,058~USD, которое вступит в силу с 19 апреля, она обязана указать, какая цена останется действительной после
          21 апреля}.
          В случае отсутствия подобных указаний со стороны Компании"=поставщика, по умолчанию действительной остается
          более низкая цена.
    \item В случае если Компания"=клиент согласна принять уведомление об удалении или повышении цены на подкод
          либо полную замену    прайс"=листа ранее чем за \textbf{7 (семь) дней}, Компания"=поставщик имеет право выслать данные
          изменения после получения письменного подтверждения от менеджера Компании"=клиента.
    \item В случаях когда изменения тарифов или кодов должны вступить в силу не ранее, чем через
          \textbf{7 (семь) дней} с момента уведомления, первый днем считается день отправки уведомления.
    \item Если Компанией"=поставщиком соблюдены все вышеозначенные вступления тарифов в силу,
          а также соблюдены условия раздела \ref{sec:confirmation-ru} данного Приложения
          (\flqq{}\nameref{sec:confirmation-ru}\frqq{}), тарифы не подлежат оспариванию Компанией"=клиентом
          после даты вступления их в действие.
  \end{enumerate}
  

\section{Подтверждение уведомлений} \label{sec:confirmation-ru}
  \begin{enumerate}[label=\thesection.\arabic*.]
   \item Уведомление об изменении цены должно быть подтверждено Стороной"=адресатом уведомления.
         В случае отсутствия подтверждения уведомления Сторона"=отправитель обязуется высылать его
         до тех пор, пока не придёт подтверждение о получении.
    \item  Сторона"=отправитель обязана добиться того, чтобы Сторона"=адресат получила уведомление
           и подтвердила его.
    \item В случае если Компания"=клиент не подтвердила получение уведомления до начала его действия,
          коды, на которые повышается тариф, и новые коды должны быть заблокированы до тех пор,
          пока Компания"=клиент не подтвердит получение уведомления.
    \item В случае возникновения расхождений, связанных с ценой, и при этом у Компании"=поставщика
          отсутствует подтверждение от Компании"=клиента на данный код, расхождение считается
          случившемся по вине Компании"=поставщика и не может ею оспариваться.
    \item Коды, на которые цена понижается, вступают в силу независимо от того, пришло подтверждение
          о получении уведомления или нет.
    \item При подтверждении изменения тарифов Компанией"=клиентом данные тарифы не подлежат оспариванию
          после даты вступления их в действие.
  \end{enumerate}

\section{Технические префиксы и тарифные планы}

  \begin{enumerate}[label=\thesection.\arabic*.]
    \item Если Компания"=поставщик предоставляет различные тарифные планы,
          которые отличаются техническим префиксом (\textit{например, стандартный прайс"=лист
          с префиксом \#11 и премиум прайс"=лист с префиксом 0647}), то Компания"=поставщик
          \textbf{обязана} указывать технический префикс в
          каждом уведомлении. В противном случае прайс"=лист не будет принят ни
          для одного из тарифных планов, а уведомление будет считаться недействительным.
    \item Уведомления, где технический префикс не указан, будут приняты для тарифного плана
          \flqq{}без префикса\frqq{}, если только использование какого"=либо технического префикса
          по умолчанию не предусмотрено Компанией"=поставщиком в договоре или техформе.
  \end{enumerate}

  \section{Изменение кодов международной нумерации}
    \begin{enumerate}[label=\thesection.\arabic*.]
     \item При смене кодов в международном формате (\textit{например, код Казахстана изменен с 73 на 77})
           Компания"=поставщик обязана прислать уведомление, где будет указано закрытие старых кодов и открытие новых.
           В противном случае никакие изменения в биллинге Компании"=клиента по умолчанию произведены не будут.
    \end{enumerate}

\thispagestyle{appendix-nofooter}

\noindent
\dotfill
\noindent
\vfill
\noindent
\parbox[t]{0.5\linewidth}{
Компания:\hfill\textbf{\ipxpname}\hspace*{0.5cm}\\
ФИО:\hfill\textit{Владимир Динкевич}\hspace*{0.5cm}\\
Должность:\hfill\textit{Директор}\hspace*{0.5cm}\\
Подпись: \hrulefill\hspace*{0.5cm}\\
Дата: \hrulefill\hspace*{0.5cm}\\
}
\hfill
\parbox[t]{0.5\linewidth}{
\hspace*{0.5cm}Компания: \TextField[name=clntname,charsize=10pt,width=62mm,height=1em,align=2,borderwidth=0,bordercolor={1 1 1}]{}\\
\hspace*{0.5cm}ФИО: \TextField[bordersep=1,name=clntattorneyprintname,charsize=10pt,width=69mm,height=1em,align=2,borderwidth=0,bordercolor={1 1 1}]{}\\
\hspace*{0.5cm}Должность: \TextField[name=clntattorneyposition,charsize=10pt,width=60mm,height=1em,align=2,borderwidth=0,bordercolor={1 1 1}]{}\\
\hspace*{0.5cm}Подпись: \hrulefill\\
\hspace*{0.5cm}Дата: \hrulefill\\
}



\end{Form}
\end{document}
