\documentclass[pdftex,a4paper,10pt,oneside]{scrartcl}
\usepackage[left=2cm,right=2cm,top=2cm,bottom=3cm,bindingoffset=0cm]{geometry}
\usepackage{cmap} % чтобы работал поиск по PDF
\usepackage[T2A]{fontenc}
\usepackage[utf8]{inputenc}

%\usepackage{lastpage}

% our vars
\def\ipxpname{IPXP Europe s.r.o}
\def\ipxpaddress{Bubenec, Ceskoslovenske armady 371/11, PSC 160 00, Prague 6, Czech Republic}

% do not enlarge space after punctuation marks
\frenchspacing
% make bold font better
\renewcommand{\bfdefault}{b}
% we can hyphen and left two symbols in Russian
\righthyphenmin=2

% use cm-super
%\renewcommand{\rmdefault}{cmr}
%\renewcommand{\sfdefault}{cmss}
%\renewcommand{\ttdefault}{cmtt}

% use pscyr
\usepackage{pscyr}
%\renewcommand{\rmdefault}{fjn}
%\renewcommand{\sfdefault}{fma}
%\renewcommand{\ttdefault}{fcr}
\renewcommand{\rmdefault}{ftm}
\renewcommand{\sfdefault}{far}
\renewcommand{\ttdefault}{fcr}
% font for PDF forms
%\renewcommand*{\familydefault}{cmr}
%\renewcommand*{\familydefault}{fjn}
\renewcommand*{\familydefault}{ftm}

% titlesec etc.
\usepackage[bf,small,compact]{titlesec}
\usepackage{indentfirst} % first paragraph indent
% format of section: X.
\makeatletter
\renewcommand*{\@seccntformat}[1]{%
   \csname the#1\endcsname.\;
}
% dot at the TOC
\renewcommand\numberline[1]{\hb@xt@\@tempdima{#1.\hfil}}
\makeatother

\lccode`\-=`\-
\defaulthyphenchar=127


\usepackage[english,russian]{babel}
% for date formatting
\usepackage{datetime}
\renewcommand{\timeseparator}{:}
\renewcommand{\dateseparator}{.}
% end datetime
\usepackage{url} % urling
\usepackage{fancyhdr} % управление колонтитулами
\usepackage{amssymb} % математические символы. пригодятся в пакете ntheorem
\usepackage{amsmath} % математические символы. пригодятся в пакете ntheorem
\usepackage[amsthm,amsmath,thmmarks,hyperref]{ntheorem} % поддержка точек после номера теорем
\usepackage[warn]{mathtext}
\usepackage{srcltx}
\usepackage{textcomp}
\usepackage{wrapfig}
\usepackage{tabularx}
\usepackage{afterpage}
\usepackage{ccaption} % подписи к рисункам с точкой и пр.
\usepackage{color}
\usepackage{makeidx}
\usepackage{xspace}
\usepackage{wasysym}
\usepackage{enumerate}
\usepackage{enumitem}    % Enhancing the list environments
\usepackage{underscore}
\usepackage{bigstrut}

\usepackage{calc}

\usepackage[pdftex]{graphicx}
\usepackage{epstopdf}
\pdfcompresslevel=9 % сжимать PDF

\newcommand{\bs}{\symbol{'134}}
\newcommand{\grad}{\ensuremath{{}^{\circ}}\xspace}

\deffootnote[2.5em]{1.5em}{1em}{\textsuperscript{\thefootnotemark}}




\addto\captionsrussian{\def\contentsname{\textbf{Оглавление.}}}
\addto\captionsrussian{\def\refname{\textbf{Литература.}}}
                        % Наклонные русские буквы в формулах при установленном
                        % пакете pscyr
                        % подпись к рисунку в виде цифры с точкой (и пробелом!),
                        % а не сдвоеточием требует пакета ccaption

\captiondelim{. }
                        % Добавить это перед begin{thebibliography} для
                        % нумерации списка литературы в виде цифры с точкой
\makeatletter
\renewcommand{\@biblabel}[1]{#1.}
\makeatother
                        % определение теорем, необходим пакет ntheorem
\newtheorem{theo}{Теорема}
                        % Конец преамбулы

% enumerate style

\renewcommand\theenumi{\arabic{enumi}}
\renewcommand\labelenumi{\theenumi.}
\renewcommand\theenumii{\arabic{enumii}}
\renewcommand\labelenumii{\theenumi.\theenumii.}
\renewcommand\theenumiii{\arabic{enumiii}}
\renewcommand\labelenumiii{\labelenumii\theenumiii.}
\renewcommand\theenumiv{\arabic{enumiv}}
\renewcommand\labelenumiv{\labelenumiii\theenumiv.}

\makeatletter \renewcommand\p@enumii{\theenumi.} \makeatother

\clubpenalty=1000
\widowpenalty=1000
\tolerance=300

% bullets
\renewcommand{\labelitemi}{$\bullet$}
\renewcommand{\labelitemii}{$\cdot$}
\renewcommand{\labelitemiii}{$\diamond$}
\renewcommand{\labelitemiv}{$\ast$}

%%%%%%%%%%%%%%%%%%%%%%%%%%%%%%%%%%%%%%%%%%%%%%%%%%
% fancyhdr
\usepackage{fancyhdr}
\pagestyle{fancy}

\newsavebox{\ipxplogo}
\savebox{\ipxplogo}{ \includegraphics[keepaspectratio]{logo-ipxp}}

\renewcommand{\headrulewidth}{0.4pt}
\renewcommand{\footrulewidth}{0.4pt}

\lhead{}
\chead{\thepage}
\rhead{\textbf{Приложение A}}

\cfoot{\usebox{\ipxplogo}}
\lfoot{\parbox{5.5cm}{\rule{5.5cm}{0.4pt}\\ \TextField[name=ipxpname,charsize=10pt,width=5.5cm,height=1em,align=1,borderwidth=0,bordercolor={1 1 1},value=\ipxpname,readonly=true]{}}}
\rfoot{\parbox{5.5cm}{\rule{5.5cm}{0.4pt}\\ \TextField[name=clntname,charsize=10pt,width=5.5cm,height=1em,align=1,borderwidth=0,bordercolor={1 1 1}]{}}}

% for the first page
\fancypagestyle{appendix}{%
\fancyhf{} % clear all header and footer fields
\chead{\thepage}
\rhead{\textbf{Приложение А}}
\lfoot{}
\lfoot{\parbox{5.5cm}{\rule{5.5cm}{0.4pt}\\ \TextField[name=ipxpname,charsize=10pt,width=5.5cm,height=1em,align=1,borderwidth=0,bordercolor={1 1 1},value=\ipxpname,readonly=true]{}}}
\cfoot{\usebox{\ipxplogo}}
\rfoot{\parbox{5.5cm}{\rule{5.5cm}{0.4pt}\\ \TextField[name=clntname,charsize=10pt,width=5.5cm,height=1em,align=1,borderwidth=0,bordercolor={1 1 1}]{}}}
}

% for the first page
\fancypagestyle{appendix-noheader}{%
\fancyhf{} % clear all header and footer fields
\renewcommand{\headrulewidth}{0.0pt}
\lfoot{\parbox{5.5cm}{\rule{5.5cm}{0.4pt}\\ \TextField[name=ipxpname,charsize=10pt,width=5.5cm,height=1em,align=1,borderwidth=0,bordercolor={1 1 1},value=\ipxpname,readonly=true]{}}}
\cfoot{\usebox{\ipxplogo}}
\rfoot{\parbox{5.5cm}{\rule{5.5cm}{0.4pt}\\ \TextField[name=clntname,charsize=10pt,width=5.5cm,height=1em,align=1,borderwidth=0,bordercolor={1 1 1}]{}}}
}

\fancypagestyle{appendix-nofooter}{%
\fancyhf{} % clear all header and footer fields
\cfoot{\usebox{\ipxplogo}}
}


\thispagestyle{appendix-noheader}
% end fancdyhdr
%%%%%%%%%%%%%%%%%%%%%%%%%%%%%%%%%%%%%%%%%%%%%%%%%%


% special macro for underscores
%\def\rp{\raisebox{-1pt}{\rule{1cm}{.4pt}}\hspace{0pt}}
\def\rp{\raisebox{0pt}{\rule{0.5cm}{.4pt}}\penalty100}

\usepackage[pdftex,a4paper,final,bookmarks=true,bookmarksopen,bookmarksnumbered,plainpages=false,colorlinks=false, unicode, pdfstartview=FitH]{hyperref}
\input{Appendix_A-ru-preamble-pdfsetup-old}
% JavaScript
\usepackage[pdftex]{insdljs}

\begin{insDLJS}[docOpen]{appendixru}{JavaScript}
function docOpen()
{

   this.getField("contractdate").textFont = font.Times;
   this.getField("contractdate").userName = "Enter contract date";

   this.getField("contractnum").textFont = font.Times;
   this.getField("contractnum").userName = "Enter contract number";

   this.getField("ipxpname").textFont = font.TimesB;
   this.getField("ipxpname").userName = "Enter company name";
   
   this.getField("clntname").textFont = font.TimesB;
   this.getField("clntname").userName = "Enter company name";
   
   this.getField("clntratesemail").textFont = font.TimesB;
   this.getField("clntratesemail").userName = "Enter email for rates notifications";

   this.getField("clntratesemailnext").textFont = font.TimesB;
   this.getField("clntratesemailnext").userName = "Enter email for rates notifications";
   
   this.getField("clntattorneyprintname").textFont = font.TimesI;
   this.getField("clntattorneyprintname").userName = "Enter attorney print name";

   this.getField("clntattorneyposition").textFont = font.TimesI;
   this.getField("clntattorneyposition").userName = "Enter attorney position";

   this.getField("clntgmt").textFont = font.TimesB;
   this.getField("clntgmt").userName = "Enter GMT offsets";

   this.getField("clntgmtsummer").textFont = font.TimesB;
   this.getField("clntgmtsummer").userName = "Enter GMT offset";


}
\end{insDLJS}
\OpenAction{/S /JavaScript /JS (docOpen();)}

\def\tfnextlinewidth{145mm}
\def\tfnextlinewidthtwo{139mm}
\def\tfnextlinewidththree{167mm}
\newlength{\fieldlength}
\newcommand{\MyTextField}[2]{
  \setlength{\fieldlength}{\linewidth-\widthof{#1}-3mm}
  #1 #2
}
\newcommand{\MyTextFieldTwo}[2]{
  \setlength{\fieldlength}{\linewidth-\widthof{#1}-2mm}
  #1 #2
}




\begin{document}
\begin{Form}
\appendix
  \selectlanguage{russian}
  
  \begin{raggedleft}
    \hfill\parbox{5.8cm}{
    \textbf{ПРИЛОЖЕНИЕ~А}\\
    к соглашению о взаимных\\
    телекоммуникационных услугах\\
    от \underline{\TextField[name=contractdate,charsize=10pt,width=5em,height=1em,borderwidth=0,
    bordercolor={1 1 1},align=0]{}}~\No~\underline{\TextField[name=contractnum,charsize=10pt,width=6em,
    height=1em,borderwidth=0,bordercolor={1 1 1},align=0]{}}
    }
  \end{raggedleft}

  \vskip 2\baselineskip

  \begin{center}
    \textbf{\Large{Изменения цен и уведомления}}
  \end{center}
  
  
  \par\textbf{Компания"=поставщик}~--- компания, принимающая трафик. \textbf{Компания"=клиент}~---
  компания, направляющая трафик. Далее вместе именуются \textbf{Стороны}.

 \section{Общие принципы}\label{anx:a-ru}

     \begin{enumerate}[label=\thesection.\arabic*.]
      \item Стороны достигли договорённости о том, что ценообразование является исключительным правом
        каждой Стороны и может не совпадать с ценовой политикой рынка. 
      \item Любая из Сторон может выступить с инициативой об изменении цен или
        кодов, направив другой Стороне письменное уведомление (далее~--- \textbf{\flqq{}Уведомление\frqq{}}).
      \item В случае изменения цен на оказываемые Сторонами услуги, а также изменения
        перечня услуг Стороны обязуются направлять уведомления об этом на
        следующие электронные адреса: 
      
        \begin{itemize}
        \item для компании \textbf{\ipxpname}: \href{mailto:rates@ipxp.net}{\textbf{rates@ipxp.net}}
          \item для \underline{\TextField[name=clntname,charsize=10pt,width=19em,height=1em,borderwidth=0,bordercolor={1 1 1}]{}}:
              \underline{\TextField[name=clntratesemail,charsize=10pt,width=75mm,height=1em,borderwidth=0,
              bordercolor={1 1 1}]{}}
              \\[\medskipamount]
              \underline{\TextField[name=clntratesemailnext,charsize=10pt,width=151mm,height=1em,borderwidth=0,
              bordercolor={1 1 1}]{}}             

        \end{itemize}

      \end{enumerate}
      
\section{Содержание уведомлений}
   
    \begin{enumerate}[label=\thesection.\arabic*.]
    \item В случае какого"=либо изменения цены на любое направление Компания"=поставщик
      должна чётко указать в уведомлении изменения по кодам. Каждое уведомление
      обязано содержать: 

      \begin{itemize}
      \item Код;
      \item Название направления;
      \item Стоимость минуты;
      \item Тарификация (1/1~--- посекундная, 60/60~--- поминутная, или др.)~---
        для каждого кода, либо единая для всех кодов в данном уведомлении; 
      \item Статус (повышение цены; понижение цены; удаление кода или подкода
        из прайс"=листа, прописанного в предыдущих уведомлениях; введение нового
        кода или подкода в прайс"=лист; цена без изменения; блокирование);
      \item Дату вступления в силу~--- для каждого кода либо единую для всех
        кодов в данном уведомлении;
      \item Статус ранее предоставляемых подкодов на данный код (если они подлежат
        изменению).
      \end{itemize}

    Каждый код должен занимать отдельную строку таблицы во вложенном файле
    формата \selectlanguage{english}{Comma Separated Values (CSV), Microsoft Excel (XLS)} \selectlanguage{russian}{или}
    \selectlanguage{english}{OpenDocument Spreadsheet (ODS)}. \selectlanguage{russian}Формат CSV явлется более предпочтительным,
    так как остальные форматы имеют ограничение на количество строк (65536) и колонок (256) на одном листе.

  \item \label{enu:2.2-ru}Цена конкретного направления соотносится только с соответствующим
    ему кодом. Название направления указывается исключительно с информативной
    целью.
  \item Маршрутизация трафика происходит на все подкоды и коды, предоставляемые
    поставщиком. При этом трафик маршрутизируется на самый длинный подкод,
    предоставленный поставщиком на это направление. \textit{Например, если цена
    на код 234806 составляет 0,13~USD, а на код 23480~---
    0,09~USD, то звонок на номер 234806121212 протарифицируется
    по цене 0,13~USD. На все остальные коды 23480{*}~--- по цене
    0,09~USD}.
  \item Маршрутизация на более длинные подкоды, начинающиеся с цифры основного
    кода, но не выделенные в прайс"=листе, будет проходить на основной
    код. \textit{Например, если поставщик предоставляет код 1, но не
    указывает в прайс"=листе отдельно код на Доминиканскую республику (1809),
    то звонок на 1809121212 будет тарифицироваться по цене, указанной
    для основного кода 1}. Такие более длинные коды должны быть
    либо заблокированы для приема трафика, либо прописаны отдельным списком
    с указанием цены или статуса \textbf{\flqq{}block\frqq{}}.
  \item Если Компания"=поставщик предоставляет новый код стационарной телефонной
    сети страны (далее~---\textbf{ \flqq{}PSTN\frqq{}}) (\textit{например, 380}),
    она \textbf{обязана} в этом уведомлении указать статус мобильных кодов и городов,
    если они имеют цену, отличную от цены PSTN (выше или ниже). Если же
    цены на коды мобильных операторов (\textit{38067, 38050
    и т.д.}) не указаны в прайс"=листе, то весь трафик на них будет тарифицироваться
    по цене PSTN, так как иное не предусмотрено уведомлениями от Компании"=поставщика.
  \item Если Сторона при предоставлении кода не может обеспечить работу каких"=либо
    подкодов, такие подкоды должны быть указаны как заблокированные. Компания"=клиент
    должна обеспечить их блокировку со своей стороны и не направлять трафик
    на данные подкоды.
  \item Трафик на направления, не указанные в прайс листе, будет тарифицироваться
    по цене \textbf{10 (десять) долларов США} за минуту.
  \item Во избежание случаев разного трактования уведомления, Компания"=поставщик
    должна при изменении цены на отдельные подкоды отображать в тексте
    уведомления все ранее открытые коды для данной страны с указанием
    их статусов (повышение, понижение, без изменения). В случае отсутствия
    детализации по подкодам они остаются без изменения, сохраняя свою
    прежнюю цену, согласно более ранним уведомлениям. 
  \item Более короткие (основные) коды, вне зависимости от того, их цена выше
    или ниже, чем у подкодов, \textbf{не замещают} действие подкодов, которые продолжают
    действовать в случае отсутствия особых указаний от Компании"=поставщика.
    \textit{Например, если Компания"=поставщик присылает код 79 по цене
    0,06~USD, но ранее предоставляла подкод 7903 по цене 0,04~USD,
    то трафик на номер 7903797979 будет протарифицирован
    по 0,04~USD, так как не было уведомления от поставщика об удалении
    этого подкода, повышении цены на него или его включении в основной
    код}. 
  \item Если Компания"=поставщик подразумевает, что более короткий код замещает
    действие всех более длинных подкодов, предоставляемых ранее, она \textbf{обязана}
    подробно описать это в уведомлении во избежание двойного трактования:
    \begin{itemize}
    \item либо предоставив полный список подкодов, подлежащих удалению, с указанием
      статуса \textbf{\flqq{}de\-le\-te\frqq{}};
    \item либо указав в теле уведомления,
      что с момента вступления уведомления в силу все подкоды тарифицируются
      по цене основного кода (см.\ пункт \ref{enu:2.11-ru})
    \end{itemize}
    В случае отсутствия пояснения о статусе подкодов цена на основной код
    прописывается, а подкоды не удаляются и тарифицируются по ранее указанным ценам. 
  \item \label{enu:2.11-ru}В случае полной замены прайс"=листа Компанией"=поставщиком
    либо полной замены тарифов на определенное направление в \textbf{теле письма}
    обязательно должно быть указано, что данное уведомление \textbf{полностью
    заменяет} предыдущие по всем направлениям либо по определенному направлению,
    обозначенному кодом (\textit{например, Узбекистан 998}). Пример такого
    уведомления: 
    \begin{quote}
      \begin{center}
        \texttt{Уважаемые Коллеги!}
      \end{center}
      \begin{flushleft}
        \indent\texttt{Официально уведомляем Вас об изменении тарифов и кодов на терминацию
        трафика с 1 сентября 2009~г.\\
        \indentПросим обратить Ваше внимание, что терминация трафика будет производиться
        \textbf{только по приведенным в данном уведомлении} кодам и тарифам по направлению
        Uzbekistan (998).\\
        \indentРанее действующие тарифы и коды на терминацию трафика по этому направлению
        просим \textbf{считать недействительными}}.\\
    \end{flushleft}
  \end{quote}
    Уведомление о полной замене тарифов по какому"=либо направлению без
    указания кода этого направления, а только лишь по названию, не может
    считаться действительным в соответствии с пунктом \ref{enu:2.2-ru} настоящего
    Приложения. 
    
  \item Если обслуживание какого"=либо подкода приостанавливается и он подлежит
    удалению из прайс"=листа, Компания"=поставщик может поступить двумя
    способами: 
    
    \begin{itemize}
    \item прислать уведомление о повышении цены на данный подкод до уровня цены
      основного кода; 
    \item или прислать уведомление об удалении данного подкода и его включении
      в основной код (а также повышении цены до уровня основного кода). 
    \end{itemize}

  \item Если Компания"=поставщик присылает коды со статусом \textbf{\flqq{}без
      изменений\frqq{}}, то цены на эти коды не изменяются в системе Компании"=клиента.
    Поэтому Компания"=поставщик несет ответственность за то, чтобы цены
    на эти коды соответствовали ранее присланным.
  \item Компания"=поставщик согласна с тем, что Компания"=клиент \textbf{не несет} ответственности
    за ошибки в уведомлениях или биллинге Компании"=поставщика. Если Компания"=поставщик
    прислала неверные цены и до даты вступления в силу этих цен не прислала
    уточнения, цены считаются принятыми и верными и оспариванию или
    пересчёту не подлежат. \textit{Например, если поставщик прислал цену на код
    1 равную 0,01~USD и не указал в прайс"=листе отдельно цены
    на подкоды 1809,1767 (Доминиканская республика), но принял трафик
    на эти подкоды, он не может требовать пересчёта цены, потому что он
    не выделил эти подкоды как самостоятельные в прайс"=листе}.
  \item Компания"=клиент согласна с тем, что Компания"=поставщик \textbf{не несет} ответственности
    за неверное внесение кодов и тарифов в биллинговую систему Компании"=клиента.
  \item В случае если присланное уведомление дает возможность для двоякого
    трактования и злоупотребления ошибкой Компании"=поставщика, Компания"=клиент
    должна обратиться к Компании"=поставщику для \textbf{уточнения и получения
    дополнительных инструкций}.
  \end{enumerate}
  

\section{Сроки вступления в силу изменений}

  \begin{enumerate}[label=\thesection.\arabic*.]
    \item Стороны договорились, что уведомления о тарифах будут вступать в силу, а расчёты биллинга производиться в следующих временных зонах:
      \begin{itemize}
        \item от \textbf{\ipxpname}: \textbf{GMT+0} (в летнее время: \textbf{GMT+1})
        \item \MyTextField{от}
                {\underline{\TextField[name=clntname,charsize=10pt,width=22em,height=1em,borderwidth=0, bordercolor={1 1 1}, default=Company name]{}}:}
                 \textbf{GMT}\underline{\TextField[name=clntgmt,charsize=10pt,width=3em,maxlen=3,height=1em,borderwidth=0,bordercolor={1 1 1}]{}}
         (в летнее время: \textbf{GMT}\underline{\TextField[name=clntgmtsummer,charsize=10pt,width=3em,maxlen=3,height=1em,borderwidth=0,bordercolor={1 1 1}]{}})
      \end{itemize}
    \item Повышения вступают в силу не ранее, чем через \textbf{7 (семь) дней} с момента уведомления Компании"=клиента.
    \item Понижения вступают в силу с момента уведомления Компании"=клиента.
    \item Удаление кодов вступает в силу не ранее, чем через \textbf{7 (семь) дней} с момента уведомления Компании"=клиента.
    \item Новые коды, если они влекут за собой повышение (\textit{например, из 79 Russia mobile по цене
          0,050~USD выделен новый код 7954 по цене 0,580~USD}), вступают в силу
          не ранее, чем через \textbf{7 (семь) дней} с момента уведомления Компании"=клиента.
          В случае если новые коды не влекут за собой повышение, они вступают в действие с момента уведомления Компании"=клиента.
    \item Полная замена прайс"=листа либо полная замена всех тарифов на определенное
          направление вступает в силу не ранее, чем через \textbf{7 (семь) дней} с момента уведомления
          Компании"=клиента. Уведомление, помеченное как полная замена прайс листа либо полная
          замена всех тарифов на определенное направление, должно иметь единую дату вступления в действие.
    \item Если Компания"=поставщик присылает изменения, вступающие в силу через \textbf{7 (семь) дней},
          то при изменении цен на эти же коды повторно в течение этих же \textbf{7 (семи) дней}
          Компания"=поставщик обязана дополнительно указать, которая из цен останется действительной по истечении
          \textbf{7 (семи) дней}. \textit{Например, если Компания"=поставщик прислала повышение на код 9989 с 0,060
          до 0,063~USD, которое вступит в силу с 21 апреля, а 19 апреля прислала понижение на код 9989 с 0,060 до
          0,058~USD, которое вступит в силу с 19 апреля, она обязана указать, какая цена останется действительной после
          21 апреля}.
          В случае отсутствия подобных указаний со стороны Компании"=поставщика, по умолчанию действительной остается
          более низкая цена.
    \item В случае если Компания"=клиент согласна принять уведомление об удалении или повышении цены на подкод
          либо полную замену    прайс"=листа ранее чем за \textbf{7 (семь) дней}, Компания"=поставщик имеет право выслать данные
          изменения после получения письменного подтверждения от менеджера Компании"=клиента.
    \item В случаях когда изменения тарифов или кодов должны вступить в силу не ранее, чем через
          \textbf{7 (семь) дней} с момента уведомления, первый днем считается день отправки уведомления.
    \item Если Компанией"=поставщиком соблюдены все вышеозначенные вступления тарифов в силу,
          а также соблюдены условия раздела \ref{sec:confirmation-ru} данного Приложения
          (\flqq{}\nameref{sec:confirmation-ru}\frqq{}), тарифы не подлежат оспариванию Компанией"=клиентом
          после даты вступления их в действие.
  \end{enumerate}
  

\section{Подтверждение уведомлений} \label{sec:confirmation-ru}
  \begin{enumerate}[label=\thesection.\arabic*.]
   \item Уведомление об изменении цены должно быть подтверждено Стороной"=адресатом уведомления.
         В случае отсутствия подтверждения уведомления Сторона"=отправитель обязуется высылать его
         до тех пор, пока не придёт подтверждение о получении.
    \item  Сторона"=отправитель обязана добиться того, чтобы Сторона"=адресат получила уведомление
           и подтвердила его.
    \item В случае если Компания"=клиент не подтвердила получение уведомления до начала его действия,
          коды, на которые повышается тариф, и новые коды должны быть заблокированы до тех пор,
          пока Компания"=клиент не подтвердит получение уведомления.
    \item В случае возникновения расхождений, связанных с ценой, и при этом у Компании"=поставщика
          отсутствует подтверждение от Компании"=клиента на данный код, расхождение считается
          случившемся по вине Компании"=поставщика и не может ею оспариваться.
    \item Коды, на которые цена понижается, вступают в силу независимо от того, пришло подтверждение
          о получении уведомления или нет.
    \item При подтверждении изменения тарифов Компанией"=клиентом данные тарифы не подлежат оспариванию
          после даты вступления их в действие.
  \end{enumerate}

\section{Технические префиксы и тарифные планы}

  \begin{enumerate}[label=\thesection.\arabic*.]
    \item Если Компания"=поставщик предоставляет различные тарифные планы,
          которые отличаются техническим префиксом (\textit{например, стандартный прайс"=лист
          с префиксом \#11 и премиум прайс"=лист с префиксом 0647}), то Компания"=поставщик
          \textbf{обязана} указывать технический префикс в
          каждом уведомлении. В противном случае прайс"=лист не будет принят ни
          для одного из тарифных планов, а уведомление будет считаться недействительным.
    \item Уведомления, где технический префикс не указан, будут приняты для тарифного плана
          \flqq{}без префикса\frqq{}, если только использование какого"=либо технического префикса
          по умолчанию не предусмотрено Компанией"=поставщиком в договоре или техформе.
  \end{enumerate}

  \section{Изменение кодов международной нумерации}
    \begin{enumerate}[label=\thesection.\arabic*.]
     \item При смене кодов в международном формате (\textit{например, код Казахстана изменен с 73 на 77})
           Компания"=поставщик обязана прислать уведомление, где будет указано закрытие старых кодов и открытие новых.
           В противном случае никакие изменения в биллинге Компании"=клиента по умолчанию произведены не будут.
    \end{enumerate}
    
\thispagestyle{appendix-nofooter}

\noindent
\dotfill
\noindent
\vfill
\noindent
\parbox[t]{0.5\linewidth}{
Компания:\hfill\textbf{\ipxpname}\hspace*{0.5cm}\\
ФИО:\hfill\textit{Владимир Динкевич}\hspace*{0.5cm}\\
Должность:\hfill\textit{Директор}\hspace*{0.5cm}\\
Подпись: \hrulefill\hspace*{0.5cm}\\
Дата: \hrulefill\hspace*{0.5cm}\\
}
\hfill
\parbox[t]{0.5\linewidth}{
\hspace*{0.5cm}Компания: \TextField[name=clntname,charsize=10pt,width=62mm,height=1em,align=2,borderwidth=0,bordercolor={1 1 1}]{}\\
\hspace*{0.5cm}ФИО: \TextField[bordersep=1,name=clntattorneyprintname,charsize=10pt,width=69mm,height=1em,align=2,borderwidth=0,bordercolor={1 1 1}]{}\\
\hspace*{0.5cm}Должность: \TextField[name=clntattorneyposition,charsize=10pt,width=60mm,height=1em,align=2,borderwidth=0,bordercolor={1 1 1}]{}\\
\hspace*{0.5cm}Подпись: \hrulefill\\
\hspace*{0.5cm}Дата: \hrulefill\\
}



\end{Form}
\end{document}
