\input{Appendix_A-en-preamble}
\begin{document}
\begin{Form}
\appendix
\selectlanguage{english}


  \begin{raggedleft}
    \hfill\parbox{5.8cm}{
    \textbf{APPENDIX~A}\\
    to  VoIP telecommunications\\
    services agreement\\
    dated \underline{\TextField[name=contractdate,charsize=10pt,width=5em,height=1em,borderwidth=0,
    bordercolor={1 1 1},align=0]{}}~\textnumero~\underline{\TextField[name=contractnum,charsize=10pt,width=6em,
    height=1em,borderwidth=0,bordercolor={1 1 1},align=0]{}}
    }
  \end{raggedleft}

  \vskip 2\baselineskip

  \begin{center}
   \textbf{\Large{Pricing and Notices}}\label{anx:a-en}
  \end{center}

  \textbf{Supplier }is a company that receives traffic. \textbf{Client
  }is a company that sends traffic. Supplier and Client are collectively
  referred to as \textbf{Parties}.   

  \section{General issues}

   \begin{enumerate}[label=\thesection.\arabic*.]
    \item Parties agree that each holds an exclusive right to set rates and
      that rate formation is independent from and may differ from the market
      pricing policy.
    \item Any Party has the right to initiate rate or codes modification by
      sending a written notification to the other Party (hereinafter \flqq{}\textbf{Rate
      Notification}\frqq{}). 
    \item In case of rate changes for services provided by the Parties as well
      as changes in the list of services the Parties agree to send relevant
      notifications to the following e-mails:
      
      \begin{itemize}
      \item for \textbf{\ipxpname}: \href{mailto:rates@ipxp.net}{\textbf{rates@ipxp.net}}
      \item for \underline{\TextField[name=clntname,charsize=10pt,width=19em,height=1em,borderwidth=0,bordercolor={1 1 1}]{}}:
              \underline{\TextField[name=clntratesemail,charsize=10pt,width=76mm,height=1em,borderwidth=0,
              bordercolor={1 1 1}]{}}
              \\[\medskipamount]
              \underline{\TextField[name=clntratesemailnext,charsize=10pt,width=151mm,height=1em,borderwidth=0,
              bordercolor={1 1 1}]{}}                          
      \end{itemize}

    \end{enumerate}

  \section{Notification contents}
    
   \begin{enumerate}[label=\thesection.\arabic*.]
    \item In case of any route rate changes the Supplier is obliged to clearly
      indicate the exact alteration of codes in the notification, such as:
    
      \begin{itemize}
      \item Code;
      \item Route name;
      \item Rate per minute;
      \item Increment (1/1~--- 1 second increments, 60/60~--- sixty second increments
        or other)~--- for each code individually or all codes in total;
      \item Status (rate increase; rate decrease; elimination from the price list
        of a code or subcode specified in the previous notices; introduction
        of a new code or subcode to the price list; rate without change; blocked
        code);
      \item Effective date~--- for each code individually or all codes in total;
      \item Status of the before provided subcodes of this code (if they are subject
        to changes).
      \end{itemize}
      

      Each code must be written in a separate table cell of an attached file
      created in Comma Separated Values (CSV), Microsoft Excel (XLS) or
      OpenDocument Spreadsheet (ODS) format. CSV is preferred one due to limited
      number of rows (65536) and columns (256) in each sheet of other formats.

    \item  \label{enu:2.2-en}The rate for the specific route should correlates with its
      appropriate code only.
      The route name is given for information purposes only. 
    \item The traffic is routed to all the codes and subcodes provided by the
      Supplier.
      In this case the traffic is routed to the longest subcode provided by the
      Supplier for this destination. \textit{For example, if the rate for code 234806
      is 0.13~USD, and for code 23480 is 0.09~USD, the call on number
      234806121212 will be billed at 0.13~USD.
      On all other codes beginning with code 23480{*}~--- at 0.09~USD}.
    \item Traffic to longer subcodes beginning with the basic code which are
      not specified in the price list, will be routed to the basic code.
      \textit{For example, if the Supplier provided the code 1, but did not
      prescribe separately the code of Dominican Republic (1809), the call on
      number 1809121212 will be billed at the rate indicated for the
      basic code 1}. Such longer codes must be either blocked for
      traffic reception or specified separately together with an indication of
      their rates or status \textbf{\flqq{}block\frqq{}}.
    \item If the Supplier provides a new fix\-ed-li\-ne country code
      (hereinafter~--- \textbf{\flqq{}PSTN\frqq{}}) (\textit{for example,
      380}), he is \textbf{obliged} to specify the status of mobile and
      co\-un\-try-ci\-ty codes in the same notice, unless their rates are not
      different (higher or lower) from the PSTN rate. If the rates for mobile
      codes  (\textit{38067, 38050 etc.}) are not indicated separately
      in the price list, all the traffic there will be billed at the rate of
      PSTN, because the Supplier's notifications do not specify other cases. 
    \item If the Supplier provides a code with certain non-op\-er\-at\-ing subcodes
      he must clearly indicate such subcodes as blocked ones. The Client should 
      block these subcodes on his side and should not route traffic there. 
    \item Any traffic sent to a destination without a rate indication in the
      current price list will be subject to a lump-ra\-te of \textbf{10 (ten) U.S.~dollars/min}.
    \item To avoid any possibility of misunderstanding and misinterpretaton of
      the Rate Notification the Supplier is obliged to restate in the Rate
      Notification all the previously provided codes for the relevant country
      along with their status indication (increase, decrease, current, etc.).
      In case the notifying Party does not restate the subcodes' status they
      shall be charged at the rates quoted in previous rate notifications. 
    \item Shorter (basic) codes, no matter their rate is higher or lower than
      for subcodes, \textbf{do not substitute} the action of subcodes which continue to
      operate unless otherwise stated in a special notice by the Supplier. 
      \textit{For example, if the Supplier sends code 79 at 0.06~USD, and
      earlier he opened subcode 7903 at 0.04~USD, traffic on number
      7903797979 will be billed at 0.04~USD, because there was
      no notification from the Supplier about deleting of this subcode,
      rate increase or its integration into a shorter (basic) code}.
    \item If the Supplier implies that the shorter code substitutes the 
      effect of all longer subcodes provided eariler, he \textbf{must} specify it
      clearly in the Notice to avoid double meaning: 
      \begin{itemize}
      \item by providing a list of subcodes which are subject to deletion and indicating the \textbf{\flqq{}delete\frqq{}} status;
      \item or by indicating in the letter body that from the moment the Notice
        enters into force all subcodes will be billed at the rate of the basic
        code (see paragraph \ref{enu:2.11-en})
      \end{itemize}
      In case the status of subcodes is not specified the basic code is
      downloaded, but the subcodes are not deleted and thus billed on the basis
      of before indicated rates.
    \item  \label{enu:2.11-en}In case of full replacement of the price list the Supplier must
      indicate in the \textbf{letter body} that current notification will \textbf{completely
      replace} the rates for either all destinations offered before or a certain
      destination within the country dialing code (\textit{for example, Uzbekistan 998}).
      Example of such notice: 
    \begin{quote}
      \begin{center}
        \texttt{Dear Colleagues,}
      \end{center}
      \begin{flushleft}
        \texttt{Officially we inform you about the change of rates and codes on traffic
        termination from September, 1, 2009.\\
        Please, pay attention that traffic termination will be accepted \textbf{only to
        the pointed in this notification codes} and rates for the destination of
        Uzbekistan (998).\\
        Previous rates and codes to these destinations should be \textbf{considered
        invalid}.}\\
    \end{flushleft}
  \end{quote}
  
    Notification about complete replacement of rates for a certain destination
    without specifying its code is considered invalid according to Paragraph \ref{enu:2.2-en}
    of the present Appendix. 

  \item If a certain subcode becomes non-op\-er\-at\-ion\-al and is to be removed from
    the price list, the Supplier can choose between 2 options: 
    \begin{itemize}
    \item to send a Notice about rate increase for this subcode to the basic code rate; 
    \item or to send a Notice about closing of this subcode and its integration into a
    basic code (and also increase of the rate to the basic code rate). 
    \end{itemize}

  \item If the Supplier sends codes with \textbf{\flqq{}no change\frqq{}} status,
    the rates for these codes remain unchanged in the Client's billing system. 
    Therefore, the Supplier takes responsibility to ensure that the rates for
    these codes correspond with the rates sent earlier.

  \item The Supplier agrees that the Client \textbf{is not} responsible for errors in
    the Supplier's notifications or billing system. If the Supplier sent 
    incorrect rates and did not send clarification before the new date came
    into effect, the rates are considered valid and accepted and therefore
    are not subject to dispute or recount.
    \textit{For example, if the Supplier sent a rate for code 1 equal to
    0.01~USD without specifying prices for subcodes 1809, 1767
    (Dominican Republic) in the price list, but accepted traffic to these subcodes,
    he can not demand a rate recount since he did not indicate these subcodes in
    the price list as separate}.
   \item The Client agrees that the Supplier \textbf{is not} responsible for the errors
    in rates and codes uploading into the Client's billing system.
  \item In case the error in the Supplier's notification gives the possibility
    of double interpretaion and misuse, the Client should ask the Supplier
    for \textbf{clarification and additional instructions}.

  \end{enumerate}


\section{Effective dates}

  \begin{enumerate}[label=\thesection.\arabic*.]
  \item The Parties agree that rate notifications will come into effect and invoices will be generated in the following time zones:
      \begin{itemize}
        \item from \textbf{\ipxpname}: \textbf{GMT+0} (summer time: \textbf{GMT+1})
        \item \MyTextField{from}
                {\underline{\TextField[name=clntname,charsize=10pt,width=22em,height=1em,borderwidth=0, bordercolor={1 1 1}, default=Company name]{}}:}
                 \textbf{GMT}\underline{\TextField[name=clntgmt,charsize=10pt,width=3em,maxlen=3,height=1em,borderwidth=0,bordercolor={1 1 1}]{}}
         (summer time: \textbf{GMT}\underline{\TextField[name=clntgmtsummer,charsize=10pt,width=3em,maxlen=3,height=1em,borderwidth=0,bordercolor={1 1 1}]{}})
      \end{itemize}
  \item Increases become effective no earlier than \textbf{7 (seven) days} from the moment of notification of the Client.
  \item Decreases become effective immediately from the moment of notifying the Client.
  \item Code deleting becomes effective no earlier than \textbf{7 (seven) days} from the moment of notification of the Client.
  \item New codes, if they involve rate increase (\textit{for example, a new code 7954
    at the rate 0.580~USD is separated from code 79 Russia mobile at the rate 0.050~USD}),
    become effective no earlier than \textbf{7~days} from the moment of notification of the Client.
    If new codes do not involve any rate increase, they become effective from the moment of notification of the Client.
  \item A complete replacement of the price list or rates for a particular destination
        becomes effective no earlier than \textbf{7 (seven) days} from the moment of notification of the Client.
        A notification, marked as a full price list or rates replacement for a particular
        destination must have a single effective date.
  \item If the Supplier notifies the Client about changes due to become effective in
        \textbf{7 (seven) days}, and afterwards is willing to change rates for the same codes within
        the mentioned period of \textbf{7 (seven) days}, the Supplier shall indicate which rate shall
        be considered valid after the \textbf{7 (seven) days} notification period. \textit{For example,
        the Supplier increased the rate for code 9989 from 0.060~USD
        to 0.063~USD and the change is due to become effective on April 21.
        Afterwards he decreased the rate for code 9989 from 0.060~USD to 0.058~USD
        effective on April 19. In this case the Supplier must indicate which rate will be considered
        valid after April 21}.
        If the Supplier fails to provide such information, the lower rate will remain in effect
  \item In case the Customer agrees to accept the Notice about removing a subcode or increasing its
        rate or a full price list replacement earlier than in \textbf{7 (seven) days}, the Supplier upon the receipt
        of a written confirmation from the Customer's manager has a right to send a notice with the
        corresponding changes.
  \item In cases when rate or code changes should become effective no earlier than \textbf{7 (seven) days}
       from the moment of notification of the Client, the day when the rate notification is sent
        by the Supplier is considered the first day of the notification period.
  \item In case all aforesaid terms of effectiveness as well as conditions of Section \ref{sec:confirmation-en}
        of the present Appendix (\flqq{}\nameref{sec:confirmation-en}\frqq{}) are observed by the Supplier,
        the rates are not subject to dispute after the date they come into force.
  \end{enumerate}


  \section{Confirmation of rate notifications} \label{sec:confirmation-en}
    \begin{enumerate}[label=\thesection.\arabic*.]
      \item The Rate Notification receipt must be confirmed by the receiving Party.
           Otherwise the notifying Party is obliged to continue sending the Rate Notification
           until such confirmation is received.
      \item The notifying Party is obliged to ensure that the receiving Party receives and confirms
           the receipt of the Rate Notification
      \item If the Client failed to provide confirmation of the Rate Notification receipt before
            the effective date, codes with increased rates and new codes must be blocked until
            the Client confirms the receipt.
      \item In case of rate discrepancies whereas the Supplier did not receive the relevant
            confirmation from the Client the discrepancy is deemed as occurred on the Supplier
            fault and is not subject to dispute.
      \item The codes with a rate decrease come into force regardless of the fact whether
            the Rate Notification confirmation was received or not.
      \item In case of confirmation of rate notification by the Client these rates are not
            subject to dispute after the date they come into force.
    \end{enumerate}
    
  \section{Technical prefixes and tariff plans}

  \begin{enumerate}[label=\thesection.\arabic*.]
     \item If Supplier provides different tariff plans which differ by technical
          prefix (\textit{for example, standard price list with prefix \#11 and premium
          price list with prefix 0647}), Supplier \textbf{must} specify the
          technical prefix in each rate notification. Otherwise the price list will
          not be accepted for any tariff plan and notification will be considered invalid.
      \item Notifications where technical prefix is not specified will be accepted as the
            \flqq{}no prefix\frqq{} tariff plan, unless the use of a certain default prefix
            is provided by the Supplier in the agreement or technical form.
    \end{enumerate}

  \section{International numbering plan change}
    \begin{enumerate}[label=\thesection.\arabic*.]
     \item When international dialing codes are changed (\textit{for example, Kazakhstan co\-un\-try-co\-de is
           changed from 73 to 77}), the Supplier must send a notification indicating
           the closure of old codes and introduction of new ones. Otherwise no default change will
           be performed in the Client's billing system.
    \end{enumerate}

\thispagestyle{appendix-nofooter}

\noindent
\dotfill
\noindent
\vfill
\noindent
\parbox[t]{0.5\linewidth}{
Company: \hfill\textbf{\ipxpname}\hspace*{0.5cm}\\
Print name:\hfill\textit{Volodymyr Dinkevych}\hspace*{0.5cm}\\
Position:\hfill\textit{Director}\hspace*{0.5cm}\\
Signature: \hrulefill\hspace*{0.5cm}\\
Date: \hrulefill\hspace*{0.5cm}\\
}
\hfill
\parbox[t]{0.5\linewidth}{
\hspace*{0.5cm}Company: \TextField[name=clntname,charsize=10pt,width=63mm,height=1em,align=2,borderwidth=0,bordercolor={1 1 1}]{}\\
\hspace*{0.5cm}Print name: \TextField[bordersep=1,name=clntattorneyprintname,charsize=10pt,width=61mm,height=1em,align=2,borderwidth=0,bordercolor={1 1 1}]{}\\
\hspace*{0.5cm}Position: \TextField[name=clntattorneyposition,charsize=10pt,width=65mm,height=1em,align=2,borderwidth=0,bordercolor={1 1 1}]{}\\
\hspace*{0.5cm}Signature: \hrulefill\\
\hspace*{0.5cm}Date: \hrulefill\\
}



\end{Form}
\end{document}



